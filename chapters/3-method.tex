\chapter{Method}
\label{chap:method}

We start of at the same spot we left of after the thesis. Before we go on to explain what's been done this semester, lets recap. 

We started off by just getting a drive with lots of \acrshort{das} data, and a script from \acrfull{asn} which set up us perfectly. We chose to rewrite python code to a language similar to that of MAtlab and Python. We landed on Julia, and with its broad ecosytstem it has support for all kinds of development we'd ever need. \\

We first wrote a 1:1 copy of the script, and then parallellized parts of the code until we had a version that could handle large amounts of data in parallel. We also wrote methods for being able to run a window over and 


\section{Overview}

The \acrshort{api} that's being created is called \texttt{Judas} (Julia and DAS) and is split in 3 modules as well as a seperate Utils file as shown in \ref{fig:ccuda}

\begin{figure}[h]
    \centering
    \includegraphics{figures/overview.png}
    \caption{Overview over our package Judas}
    \label{fig:judasoverview}
\end{figure}

% Data
\section{Data}

\subsection{Dataset}

For this project, we will be working with data recorded from BANE-NOR, where a train is travelling between Trondheim and Storen the 31st of august 2021 \footnote{Working with national infrastructure requires security clearance, see \ref{app:conf} for more details}. Data is split between 5000 sensor channels, lasting us a day.
The data is recorded in \acrshort{hdf5} files, with each file containing 20000 samples for 10 seconds, giving us a sample rate of $2000Hz$. The


\begin{table}[h]
\centering
\begin{tabular}{|r|r|r|r|r|}
\hline
\textbf{1}                & \textbf{2}               & \multicolumn{1}{c|}{\textbf{...}} & \textbf{n-1}             & \textbf{n}               \\ \hline
1f-3                      & 2.3f-5                   & 4f-4                              & 3.4f-6                   & 3f-1                     \\ \hline
3f-1                      & 3f-1                     & 3f-1                              & 3f-1                     & 3f-1                     \\ \hline
\multicolumn{1}{|c|}{...} & \multicolumn{1}{c|}{...} & \multicolumn{1}{c|}{...}          & \multicolumn{1}{c|}{...} & \multicolumn{1}{c|}{...} \\ \hline
4f-2                      & 3f-1                     & 3f-1                              & 3f-1                     & 3f-1                     \\ \hline
\end{tabular}
\caption{Table explaining how the table looks like}
\label{fig:datatable}
\end{table}



\begin{figure}[h]
    \centering
    \includegraphics{figures/dataflow.png}
    \caption{Dataflow from we read HDF5 files to we are ready to train}
    \label{fig:dataflow}
\end{figure}

\subsection{Pre processing}

\subsubsection{Resampling and Channel decimation}

Following our previous work \cite{projthesis}, the following improvements have been made to \texttt{Judas}: 

\begin{enumerate}
    \item Judas now make use of multiple processors contra multithreading, seeing major speedups.
    \item Function \texttt{load\_DAS\_files} now writes processed data to a single file back.
    \item Methods for resampling and chann
\end{enumerate}


\subsubsection{Window and Filter operations}

After the data has been loaded \cite{projthesis}, it's still not ready for processing. 

The function \lstinline|process_DAS_data| takes in the signal data, perform a tukey \ref{dsp:tukey} window algorithm over it. We then run a bandpass filter algorithm with Butterworth over it to extract the and 

\begin{figure}[h]
    \centering
    \lstinputlisting{code/procdasdata.jl}
    \caption{Process DAS data function}
    \label{fig:procdasdata}
\end{figure}


After applying a window function and a filter function to combat oscilleration, we're now ready to run an fft over this data. 


The 

\begin{figure}[h]
    \centering
    \includegraphics{figures/banenor_20210831.png}
    \caption{Heatmap of the processed train data during the 31st of august 2021 from Trondheim to Storen}
    \label{fig:trdstrdas}
\end{figure}

\subsection{Data preparation before Training} 

After the preprocessing, some final work must be done before we can train the model. Normalization, also called data standardization, 
is a basic staple of preprocessing, we  . This normalization makes sure all the values . For our model, we use z-score normalization, 
which normalizes vales to be in the range 0 to 1. The f

\begin{equation}
    x' = \frac{x - \mu}{x}
\end{equation}

This method is implemented as such in \texttt{Flux.jl}

\begin{figure}[h]
    \centering
    \lstinputlisting{code/norm.jl}
    \caption{Flux normalization function}
    \label{fig:normalize_data}
\end{figure}

Where $\mu$ is the mean of the data, and $\sigma$ is the standard deviation of data.

\subsubsection{Data augmentation}


% NETWORK ARCH
\section{Network Architecture}

The model consists of

\begin{itemize}
    \item Batch size 
    \item skdfj
    \item sdfk
\end{itemize}

\begin{table}[h]
    \centering
    \begin{tabular}{ l c r }
      \hline
      Layer (type) & Output shape & Param \# \\ \hline
      lstm\_1 (LSTM) & (None, 10, 256) & 1234678 \\ \hline
      lstm\_1 (LSTM) & (None, 10, 256) & 1234678 \\ \hline
      lstm\_1 (LSTM) & (None, 10, 256) & 1234678 \\ \hline
      lstm\_1 (LSTM) & (None, 10, 256) & 1234678 \\ \hline
      lstm\_1 (LSTM) & (None, 10, 256) & 1234678 \\ \hline
      Total params: & & \\
      Trainable params: & & \\ 
      Non-Trainable params: 0 & & \\ \hline
      
    \end{tabular}
    \caption{Parameters for our architecture}
    \label{tab:archparams}
\end{table}


% DAS
\section{Distributed Acoustic Sensing}
\label{back:das}

\acrfull{das} is a type of sensing technology that uses fiber optic cables to measure strain and vibrations along their length. By employing specialized optoelectronic devices, \acrshort{das} systems can detect and measure acoustic disturbances over vast distances with high spatial and temporal resolution. This technology is widely used for monitoring conditions within geophysical environments. Some of the usecases of \acrshort{das} at \acrshort{cgf} include

\begin{itemize}
    \item Movement of traffic
    \item Detection of landslides
    \item Observation of Whales outside Svalbard
    \item Seismic activity detection
\end{itemize}


\begin{figure}[!h]
    \centering
    \includegraphics[width=0.7\linewidth]{figures/das_example.png}
    \caption{DAS data frame example}
    \label{fig:dasframe-ex}
\end{figure}

\subsection{Numerical analysis}

\acrshort{das} data can be stored in several different file formats. Data stored in \acrshort{hdf5} files have the advantage of storing additional metadata beside it and is ideal for complex datasets requiring extensive contextual information. Other formats such as TDMS are also structured around hierarchical data \cite{10.1145/800196.805973}, and is often used for storing \acrshort{das} data. Regardless of the chosen format, certain metadata are crucial for effectively handling and interpreting DAS data:

\textbf{Timestamps} is used for temporal alignment and analysis

\textbf{Gauge length} is the spatial resolution of measurements

\textbf{Channel decimation} stores information on spatial sampling. Not all channels along the total measurement is stored, and so to understand the location of a signal, the gauge length in combination with the channel decimation tells us the exact distance from the start of the measurement.

\textbf{Sample Rate} is the temporal resolution of the data, and is measured in hertz.

\acrshort{das} data itself is stored as a matrix, usually where the axis represent time and signal channels.

\begin{table}[!h]
\centering
\begin{equation*}
\mathbf{X} = \begin{bmatrix}
x_{1,1} & x_{1,2} & \cdots & x_{1,n-1} & x_{1,n} \\
x_{2,1} & x_{2,2} & \cdots & x_{2,n-1} & x_{2,n} \\
\vdots & \vdots & \ddots & \vdots & \vdots \\
x_{t-1,1} & x_{t-1,2} & \cdots & x_{t-1,n-1} & x_{t-1,n} \\
x_{t,1} & x_{t,2} & \cdots & x_{t,n-1} & x_{t,n}
\end{bmatrix}
\end{equation*}
\label{fig:dasmatrix}
\end{table}

In the matrix $\textbf{X}$ above, $t$ denotes the max amount of recordings for the file; for one second this would be equal to the sample rate. $n$ is the amount of channels stored. To process this data as fast as possible, it is important to consider which memory alignment the programming language of choice uses. In Julia, MatLab and Fortran, memory is stored in column-major order, while in most other languages, memory is stored in a row-major order. 

\section{SignalProcessing.jl}

After data has been read and processed, it's ready to be processed by signal processing functions 

\section{AI.jl}

The main aspects of this code lies within the module \texttt{AI.jl}. 


\subsection{JudasNET}

\texttt{JudasNET} is the architecture of choice for interpreting data

\begin{figure}{h}
    \centering
    \includegraphics[scale=.6]{figures/judasnet.png}
    \caption{Flowchart over JudasNET}
    \label{fig:judasnet}
\end{figure}

\subsubsection{Modification to Flux.jl}

As mentioned previously, one of the main benefits with Julia packages is the ability to easily add new functionality to packages. When working with \texttt{Flux.jl}, this has certainly been useful. For our research, we had to write our own LSTM functors to be able to achieve a LSTM Autoencoder.  \\

The first of this functors is a regular LSTM, but with \texttt{relu} as the activation function for the different gates. Secondly, we rewrote the TimeDistributed layer \cite{keras} for Flux since there is no corresponding alternative as of the writeday of this paper. The final layer we had to recreate is what's known as a \texttt{RepeatVector} in Keras. When we have all of these together, we can finally put them together and write our model. 

\begin{figure}[h]
    \centering
    \lstinputlisting[language=Julia]{code/judasnet.jl}
    \caption{JudasNET flux model}
    \label{fig:fluxjudasnet}
\end{figure}