\subsection{Python and Tinygrad}
\label{back:tiny}

It is no secret that python is the \textit{de-facto standard} language when it comes to \acrshort{ai} and data science. Even though we rely mostly on Julia for our code, Python still 




\subsubsection{Python}

Python is a dynamically typed, weak language famously known for it's \textit{easy-to-learn} syntax. Created by Guido Van Rossum in the late 80-s \cite{python}, Python has slowly emerged as one of the fastest growing programming languages ever created \cite{srinath2017python}. 

Virutally all of the larger libraries for data science and \acrlong{ml} have bindings to \gls{python}, or are written in Python from scratch. Some examples include \texttt{Pandas}, \texttt{NumPy}, \texttt{SciKit Learn}, \texttt{Pytorch} and \texttt{TensorFlow}. Many of these rely on code written in C or C++ to be fast, sincc

Due to the large ecosystem already established, it's seemingly a hard task  \\

\subsubsection{Tinygrad}

