\section{Recurrent Neural Networks}

\acrfull{rnn}

The main advantage of \acrshort{rnn} 

The building blocks of an \acrshort{rnn} is the recurrant neuron. They take an input $x_t$ at time $t$, with the state at previous time step $h_{t-1}$ and produces the next time step. 

\begin{equation}
    h_t = f(W_hh_{t-1}+W_xx_t)
\end{equation}

RNN have been used witihn geophysical applications before \cite{maulik2020recurrent}. 

Typically, \acrshort{relu} is used as the activation function in these networks, as it is cheap to perform and we don't need more expressive activation functions. Other common activation functions are sigmoid and tanh. 

$$g(z) = max(0, z)$$
$$g(z) = \frac{1}{1 + e^{-z}}$$
$$g(z) = \frac{e^z - e^{-z}}{e^z + e^{-z}}$$

Although RNN has several advantages, they are highly prone to both vanishing gradients as well as exploding gradients. 

Recurrent Batch Normalization
\begin{equation} \label{eq:bnrnn}
    BN(\textbf{h}; \gamma, \beta) = \beta + \gamma \circ \frac{
    \textbf{h} - \hat{E}[\textbf{h}]
    }{
    \sqrt{\hat{Var} [\textbf{h}] + \epsilon}
    }
\end{equation}
