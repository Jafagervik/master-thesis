\section{Julia}

Created in 2009, first released in 2012 as part of a master thesis \cite{juliaMs} and further a doctural thesis \cite{juliaPHD}, Julia is a high performance, dynamically typed programming language with the goal of being a fast and performant language like C, the simplicity of Python, linear algebra extensions like matlab, and statistics like R \cite{julia}.  \\ 

Like Python, Julia can be used in a script-like fashion through its \acrfull{repl}. Notebook style programming has been a standard way for data analysts to write their code, and Jupyter notebooks support Julia as well through the \texttt{IJulia.jl} package. 

Julias fluent type system, accompanied by easy syntax, high performance, \acrshort{repl} tools makes it a great contender for data analysis. We've previously proven how Julia effectively deals with I/O operations, \cite{projthesis}.


DOI \cite{doi:10.1137/141000671}

\subsection{AI in Julia using Flux.jl}
\label{back:flux}

Flux is a machine learning library written entirely in Julia released and published in 2018 \cite{Flux.jl-2018} \cite{Innes2018}. It allows the user to create all. \acrshort{gpu} support is also native, through the inclusion of \texttt{CUDA.jl}

Other alternatives to \texttt{Flux.jl} are \texttt{MLJ.jl} and \texttt{Tensorflow.jl}, which is essentially a Julia wrapper for Tensorflow functionality. 

\subsection{Anomaly detection in Julia}

Julia has support for working on outlier detection as follows: \cite{muhr2022outlierdetectionjl}.




