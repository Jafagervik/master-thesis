\section{Unsupervised Learning}

The major bottleneck of all kinds of machine learning tecniques is data. The more diverse and varied a .

When it comes to \acrshort{ai} and \acrshort{ml} we usually differentiate between tree major types, those being supervised, unsupervised and semi-supervised learning (or self-supervised learning). They differ in the roles that can occur.


\subsection{Linear Layers}
\label{back:linear}

Fully Connected Layer, Dense Layer or Linear Layers  are types of an operation used in several deep learning models. This layer take an input vector and maps it to an output of learnable parameters. The resulting parameters are thus a weight matrix $W$ and a bias vector $b$. It is defined as follows:

\begin{align}
    Y &= XW + b
\end{align}

where:
\begin{align*}
    X & \text{ - input vector [$n x m$] where $m$ is the amount of input features and $n$ is the batch size} \\
    W & \text{ - weight matrix [$m x p$] where $p$ is the amount of output features} \\
    b & \text{ - bias vector of size $p$} \\
    Y & \text{ - output vector [$n x p$]} \\
\end{align*}

\begin{figure}[h]
    \centering
    \includegraphics[width=0.5\linewidth]{figures/linearlayer.png}
    \caption{Example of a $3x2$ linear layer}
    \label{fig:linearlayer}
\end{figure}

Furthermore, an activation function such as \gls{relu} \ref{eq:relu} is applied to the output to achieve nonlinearity.
Lets denote this function as $f$, we now get the following equation.

\begin{equation}
    Z = f(Y) = f(XW+b)
\end{equation}

Linear layers have several advantages, such as computational efficiency, flexibility as well as intrerprebility, where the weight and bias vectors can be interpreted as learned parameters. They also serve as building blocks for other components, such as \acrshort{rnn}s, \acrshort{lstm}s or even attention blocks. \\ 

However, linear layers have several limitations. Due to their inherent linearity, they are prone to overfitting and struggle to capture complex relationships in data. This can limit their ability to extract more complex features, potentially reducing some of the model's discriminative power. Furthermore, the size of linear layers can become problematic, especially in fully connected dense neural networks. Each neuraon connection between layers requires storing weights and biases, which increases the overall model size. This can create bottlenecks for many hardware accelerators, such as \acrshort{gpu}s, when dealing with large models.

\subsubsection{Early Stopping}

Overfitting is a common problem within \acrlong{ml}. It occurs when a model is trained , and fails to fit additional data. \textit{Early Stopping} is a regularization technique that aims to avoid such issue. When training an optimizer, such as \acrshort{adam} or \acrshort{sgd}, we can notice when a model is overfitting by studying the validation loss. If we notice an increase of validation loss, the early stop mechanism will stop the training altogether if no improvement of optimal validation loss is found after $p$ amount of epochs. The validation loss is commonly used to evaluate the hyperparameters of a model, and ac 


\begin{align*}
&\text{Stop at epoch } T \text{ if:} \\
&\forall i \in \{T-p+1, ..., T\}: L_t(i) > L_t^* - \epsilon \\
&\text{where } L_t^* = \min_{j=1}^{T} L_t(j) \\
\\
&\text{Given:} \\
&L_t(t) \text{ is the train loss at epoch } t \\
&p \text{ is the patience (number of epochs to wait)} \\
&\epsilon \text{ is a small threshold for improvement}
\end{align*}

\subsection{Parallelism within \acrlong{ml}}

With rapid evolving deep learning architectures, the importance of scalable model training and networks grows larger each year. It is even estimated that these networks grow 1,5x each year \cite{9499913}, making parallelization a vital topic when it comes to \acrlong{ml} to accommodate ever increasing memory needs. Several different hardware accelerators have been created to best accommodate these needs, the most apparent of these are \acrshort{gpu}s. NVIDIA have for several years dominated this market, and their hardware is becoming faster and increasing in memory. \\

By workers, we mainly refer to \acrshort{gpu}s, but this could also be processors or other types of hardware accelerators such as TPUs.


\subsubsection{Model Parallelism}

Deep learning models need to store a lot of data. Weights and biases tend to take up a lot of memory, thus requiring the need of splitting up a model across several workers. As an example, given a model $M$ of 50 layers, we can split this model in 2 parts by having a worker $A$ manage the first 25 layers, and worker $B$ manage the latter half.

The overhead of transferring data across these workers can become a bottleneck, so this should only be utilized when absolutely necessary.

\subsubsection{Data Parallelism}

Data parallelism refers to partitioning data across multiple workers.  Given a large set of data, we can split these data across the workers and store a copy of the model on each worker, calculate gradients across them all and update the trainable parameters for the model. An example of this would be to split a batch of size $b$ across $n$ workers, calculate the gradients and update the model. Before updating the parameters of the model, an average across all parameters is calculated, and each of the copies of the model is updated before continuing.

\subsubsection{Hybrid Parallelism}

This kind of parallelization is a combination of the two previously mentioned techniques. By first splitting a model across several workers, data is subsequently split across multiple workers. 

\subsection{CNN - Convolutional Neural Networks}
\label{back:cnn}

\acrfull{cnn}, like the linear layer discussed in section \ref{back:linear}, is a type of feed forward networks commonly used in deep learning. First properly introduced in 1989 \cite{NIPS1989_53c3bce6}, they rely on kernel operations such as a convolution or matrix multiplication to calculate features.    \\

\begin{figure}[h]
    \centering
    \includegraphics[width=0.8\linewidth]{figures/convolution.png}
    \caption{Example of a 2D Convolution}
    \label{fig:2dconv}
\end{figure}

A convolution is a matematical operation which can be described continiously as such:

\begin{equation}
(f * g)(t) = \int_{-\infty}^{\infty} f(\tau) g(t - \tau) d\tau
\label{eq:contconv}
\end{equation}

\begin{equation}
   (f * g)(x, y) = \sum_{m=0}^{M-1} \sum_{n=0}^{N-1} f(m, n)g(x-m, y-n) 
\label{eq:conv}
\end{equation}

Compared to linear layers, \acrshort{cnn}s are better at extracting more complex features and combating the vanishing gradient problem. Only the size of the kernel is used to stored neurons when performing cross correlation convolution, compared to linear layers where all the weights between layers needs to be stored. These types of networks are highly applicable to numerous different tasks, such as recognition, classification and language model tasks. \\

\subsubsection{Pooling}

Typically, a convolution is followed by an activation function and then a pooling operation. A pooling operation works like a filter $f$, selecting a certain value from a set of numbers based on a measurement. The three most common operations are \textit{max pool}, \textit{min pool} and \textit{avg pool}. Out of these, the \textit{max pool} operation operation is most commonly used. This operation reduces dimensionality, outputting only the most important feature for future layers. \\


\begin{figure}[h]
    \centering
    \includegraphics[scale=0.4]{figures/pooling.png}
    \caption{Example of a $2 x 2$ max pool operation with stride 2}
    \label{fig:maxpool}
\end{figure}