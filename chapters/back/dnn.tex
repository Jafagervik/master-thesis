\section{Deep Learning}

% Deep Learning
\acrfull{dl} is a subset of \acrshort{ml} and \acrshort{ai}, where the objective is to learn underlying representations of data \cite{lecun2015deep}. In \acrshort{dnn}s, neurons act as the fundamental building blocks. Layers of neurons are grouped into three categories of layers: input-, output- and \textit{hidden layers}.  

\begin{figure}[!h]
    \centering
    \includegraphics[width=0.5\linewidth]{figures/dnn.png}
    \caption{Example of a dense neural network with one input layer, two hidden layers and one output layer}
    \label{fig:densenn}
\end{figure}

We mainly differentiate between three subcategories of \acrshort{dl}:

\begin{itemize}
    \item \textit{Supervised learning}: Data is labeled, and the goal of the network is trained to make sure the outputs match these labels.
    \item \textit{Unsupervised learning}: The network learns patterns exclusively from unlabeled data and tries to learn the underlying structure of the input vectors \cite{KARHUNEN2015125}. 
    \item \textit{Semi-supervised learning}: Some data may be labeled
\end{itemize}



\subsection{Fully Connected Neural Networks}
\label{back:linear}

\acrfull{fcnn} are fundamental networks used in many deep learning models. Also called dense-, or linear networks, they are characterized by all neurons between two layers being connected. 
\begin{figure}[!h]
    \centering
    \includegraphics[width=0.5\linewidth]{figures/linearlayer.png}
    \caption{Example of a dense neural network}
    \label{fig:densenn}
\end{figure}


Linear layers perform a linear transformation on the input data, mapping an input vector to an output vector using learnable parameters. The operation can be defined as follows:
\begin{equation}\label{f:wxb}
    y = wx+b
\end{equation}
Here, $x$ is the input vector, $W$ is the weight matrix, $b$ is the bias and $y$ is the output vector.

\begin{figure}[!h]
    \centering
    \includegraphics[width=0.7\linewidth]{figures/dl.png}
    \caption{Neurons in a layer with their corresponding weights, a bias $b$ and a non-linear activation function $f$}
    \label{fig:dl}
\end{figure}

Furthermore, an activation function such as $f$ is applied to the output to achieve nonlinearity. By applying $f$ to the output $y$, we now get the equation:
\begin{equation}\label{f:fwxb}
    y = f(wx+b)
\end{equation}

Some of the more popular activation functions include \cite{szandala2021review}: 

\begin{equation}
    Relu(z) = max(0, z)
\end{equation}

Relu (Rectified Linear Unit) has been stated to be the most popular activation functions as late as 2018. Both it's forward and backward pass steps quickly, thus enabling more efficient training compared to other alternatives.

\begin{equation}
    \sigma(z) = \frac{1} {1 + e^{-z}}
\end{equation}

The sigmoid activation is very popular to its very nature of its range being $(0,1)$ compared to other functions where the range is often broader. This makes it good at outputting probibalistic values.

\begin{equation}
    tanh(x) = \frac{e^x - e^{-x}}{e^x + e^{-x}} = \frac{1 - e^{-2x}}{1 + e^{-2x}}
\end{equation}

Similar to sigmoid, the hyperbolic tangent has outputs in a relatively small range $(-1, 1)$
The major bottleneck of all kinds of machine learning tecniques is data. The more diverse and varied a . \\

Linear layers have several advantages, such as computational efficiency, flexibility as well as intrerprebility, where the weight and bias vectors can be interpreted as learned parameters. They also serve as building blocks for other components, such as \acrshort{rnn}s \cite{schmidt2019recurrent}, \acrshort{lstm}s \cite{lstm} or even more novel architectures such as the transformer\cite{vaswani2017attention}. \\ 

However, linear layers have several limitations. Due to their inherent linearity, they are prone to overfitting and struggle to capture complex relationships in data. This can limit their ability to extract more complex features, potentially reducing some of the model's discriminative power. Furthermore, the size of linear layers can become problematic, especially in \acrshort{fcnn}s. Each neuron connection between layers requires storing weights and biases, which increases the overall model size. For a smaller dense network where the layers are of size $[10, 5, 1]$, the total amount of parameters becomes:
$$\begin{aligned}
S &= (10 \times 5 + 5) + (5 \times 1 + 1) \
&= 50 + 5 + 5 + 1 \
&= 61 \text{ parameters}
\end{aligned}$$
Assuming each parameter is stored as a single-precision floating-point number (\texttt{Float32}, 4 bytes), the total memory size is:
$$\begin{aligned}
\text{Memory size} &= 61 \text{ parameters} \times 4 \text{ bytes/parameter} \
&= 244 \text{ bytes}
\end{aligned}$$ While 244 bytes is small, larger dense networks can quickly consume gigabytes of memory. This can create bottlenecks for hardware accelerators like \acrshort{gpu}s, which typically have less VRAM compared to the \acrshort{ram} available to \acrshort{cpu}s.
\subsection{Convolutional Neural Networks}
\label{back:cnn}

\acrfull{cnn}, a specialized type of feed-forward neural network, has become a cornerstone in \acrshort{dl} architectures. This architecture introduces a powerful approach to processing matrix data, especially images.
The foundations of \acrshort{cnn}s started all the way back in the 1960s, focusing on the visual cortex \cite{hubel1962receptive}, but they were first properly introduced to the \acrshort{ml} field in 1990 by Yan LeCun \cite{NIPS1989_53c3bce6}. Since then, \acrshort{cnn}s have undergone significant developments, leading to breakthroughs in various fields of artificial intelligence. \\
%
At their core, \acrshort{cnn}s rely on kernel operations, primarily convolutions, to calculate features. The two-dimensional convolution is a mathematical operation that can be formulated as follows:
\begin{equation}
   (f * g)(x, y) = \sum_{m=0}^{M-1} \sum_{n=0}^{N-1} f(m, n)g(x-m, y-n) 
\label{eq:conv}
\end{equation}
Here $f$ is the kernel, $g$ is the input matrix, $x$ and $y$ are the row and column of the input matrix, $m$ and $n$ are the row and column in the kernel, and $M$ and $N$ are the number of rows and columns in $g$ respectively. This is further illustrated in Figure \ref{fig:2dconv}.
%
\begin{figure}[!h]
    \centering
    \includegraphics[width=0.8\linewidth]{figures/convolution.png}
    \caption{2D Convolutional operation example. The dot product between the kernel $K$ and each submatrix of input $I$ is computed and added to output $O$. This specific convolution reduces dimensionality.}
    \label{fig:2dconv}
\end{figure}
\clearpage
 When performing cross-correlation convolution, only the size of the kernel is used to store neurons, compared to linear layers, where all the weights between layers need to be stored. This dramatically reduces the memory requirements compared to \acrshort{fcnn}s. The convolutional operation is essentially multiple matrix multiplication between different regions in the data and a kernel. These multiplications can be performed in a parallelized manner. Matrix multiplications have undergone significant improvement over the years \cite{karstadt2020matrix}, and have, with the introduction of CUDA, been further optimized for better-suited hardware architectures such as \acrshort{gpu}s. With the rapid improvements of \acrshort{gpu}s \cite{bell2008efficient}, the computational efficiency of both linear and convolutional layers has improved drastically. This has resulted in overall lower energy requirements for model training, reduced training times, and the introduction of distributed large-scale model training \cite{mungoli2023scalable}. 

Unlike fully connected layers that compute global interactions, convolution operations in \acrshort{cnn}s focus on local regions of data as seen in Figure \ref{fig:2dconv}. This approach allows for improved feature extraction, making them particularly effective for tasks involving spatial data, compared to \acrshort{fcnn}s. 

Furthermore, due to \acrshort{cnn}s inherent quality of feature extraction, they are not as prone to the vanishing gradient problem \cite{tan2019vanishing} as \acrshort{fcnn}s. This occurs when gradients propagated backward through the layers become very small, making it difficult for the network to update its weights effectively. \\

\begin{figure}[!h]
    \centering
    \includegraphics[scale=0.8]{figures/conv2d.jpg}
    \caption{Example of a \acrshort{cnn} by Liu et al. \cite{LIU2021193}}
    \label{fig:cnn}
\end{figure}

\subsubsection{Pooling}
%
In \acrshort{cnn}s, a convolutional layer is followed by an activation function $f$ and then a pooling operation as seen in Figure \ref{fig:2dconv}. Pooling reduces the spatial dimensions of the data, making the network more computationally efficient and helping to extract more dominant features \cite{SUN201796}.
The three most common pooling operations are max pooling, min pooling, and average pooling, with max pooling being the most widely used. Max pooling works by selecting the maximum value from a defined region of the input feature map.
\begin{figure}[!h]
\centering
\includegraphics[scale=0.4]{figures/pooling.png}
\caption{Example of a $2 \times 2$ max pooling operation with stride 2}
\label{fig:maxpool}
\end{figure}
Figure \ref{fig:maxpool} illustrates a $2 \times 2$ max pooling operation with a stride of 2. The pooling kernel moves across the input matrix $M$, selecting the maximum value from each $2 \times 2$ region it covers. This process is repeated until the entire input has been processed, resulting in reduced spatial dimensions in both height and width. 
\clearpage
\subsection{Autoencoder}

Autoencoders are specific types of neural networks used to learn efficient encodings of unlabeled data and then decode them to reconstruct the original data \cite{10.5555/104279.104293, bank2021autoencoders}. Autoencoders can be represented by two models, the encoder $E_\phi$ and the decoder $D_\theta$, where $\phi$ and $\theta$ are the parameters of the model. The relationship between these can be formulated as such: 

\begin{equation}\label{eq:enc}
E_\phi: X \rightarrow Z 
\end{equation}
\begin{equation}\label{eq:dec}
D_\theta: Z \rightarrow X
\end{equation}

$E_\phi$ compresses data $X$ into a latent representation $Z$. $D_\theta$ then decodes $Z$, thus outputting a reconstructed dataset of the same dimensions as the input. $E_\phi$ can be seen as a compressing model, while $D_\theta$ can be seen as a decompressing model. This is further showcased in Figure \ref{fig:aediagram}. 

\begin{figure}[!h]
    \centering
    \includegraphics[scale=0.4]{figures/ae.png}
    \caption{Example of a dense autoencoder architecture}
    \label{fig:aediagram}
\end{figure}

The optima for any kind of autoencoder becomes that of lossless encoding, which can be described as such:
\begin{equation}
    X' \approx D_\theta(E_\phi(X))
\end{equation}

When the model is sufficiently trained for a specific task, $D$ \textit{may} become unnecessary for certain applications such as data reconstruction and denoising \cite{vincent2010stacked}. If the primary goal is feature extraction or dimensionality reduction, $E$ alone can be used to map input data to the lower-dimensional latent space. By utilizing only $E$, the overall complexity and size of the model $M$ can be reduced, which may be beneficial in scenarios with computational or memory constraints. For other tasks, including image reconstruction \cite{7797236}, signal analysis \cite{andrysiak2016machine}, and anomaly detection \cite{bank2021autoencoders}, the entire model is often needed.
\subsubsection{Latent Space}

The latent space $Z$ in autoencoders aims to capture essential features of the input data $X$ in a lower-dimensional representation, as displayed in Figure \ref{fig:aediagram}. However, traditional autoencoders face limitations in their generative capabilities \cite{bank2021autoencoders}. While they are trained to reconstruct original data accurately, they cannot typically generate new, diverse samples from $Z$. This limitation arises because regular autoencoders do not force any specific structure on $Z$ beyond compressing $X$. As a result, the latent representations may not be continuous or meaningful generative tasks.

\subsubsection{Limitations}
In addition to the issues with regular autoencoders in regard to decoding the latent space, these autoencoders have several disadvantages. The first of these is overfitting. Secondly, the lack \textit{regularization}, which can lead to poor generalization to unseen data. Furthermore, noise in the input data may potentially lead to large changes in the latent space. 

\subsubsection{Convolutional Autoencoders}
As mentioned in Section \ref{back:linear}, dense networks can struggle with feature extraction. This is also the case for dense autoencoders. By introducing convolutional layers, the autoencoder becomes more adept at image reconstruction and denoising \cite{zhang2018better}. These networks are called convolutional autoencoders (\acrshort{cae}). Another benefit of these are the reduction of parameters, as mentioned in Section \ref{back:cnn}.


\clearpage
\subsection{Variational Autoencoder}
\label{back:vae}

The \acrfull{vae} \cite{kingma2022autoencodingvariationalbayes} is a type of autoencoder that aims to solve the lack of generative capabilities within regular autoencoders. \acrshort{vae}s are generative models that sample the latent space through a probabilistic distribution. This makes them suitable for image generation tasks \cite{vahdat2020nvae}, something regular autoencoders are unable to do due to their deterministic latent representation.

\begin{figure}[!h]
    \centering
    \includegraphics[scale=0.4]{figures/vae.png}
    \caption{Variational Autoencoder Architecture Diagram}
    \label{fig:vaediagram}
\end{figure}


\subsubsection{Reparametrization Trick}

\acrshort{vae}s can be trained efficiently using backpropagation due to a technique known as the \textit{reparameterization trick} \cite{kingma2022autoencodingvariationalbayes}. This is necessary because \acrshort{vae}s involve sampling from a stochastic latent variable $z$, which would normally hinder gradient-based optimization.

The idea is to express the sampling of $z$ from the approximate posterior $q_\phi(z|x) = \mathcal{N}(\mu, \sigma^2)$ as a deterministic function of the encoder outputs ($\mu$ and $\sigma$) and an additional noise variable $\epsilon$. Specifically:
\begin{equation}
    z = \mu + \sigma \odot \epsilon, \quad \epsilon \sim \mathcal{N}(0, I)
\end{equation}

Here, $\odot$ denotes element-wise multiplication. This formulation allows gradients to flow through the sampling process, enabling end-to-end training of the model.

The reparameterization trick transforms the optimization problem from one involving expectations over $q_\phi(z|x)$ to one involving expectations over $p(\epsilon)$, which is fixed and independent of the model parameters ($\phi$ and $\theta$):

\begin{equation}
    \mathbb{E}_{z \sim q_\phi(z|x)}[f(z)] = \mathbb{E}_{\epsilon \sim \mathcal{N}(0,I)}[f(\mu + \sigma \odot \epsilon)]
\end{equation}

This formulation makes training of \acrshort{vae} models feasable, even for gradient based optimizers, such as \acrshort{adam} or \acrshort{sgd}.




\subsubsection{Evidence Lower Bound}
In the context of \acrshort{vae}s, \acrshort{elbo} is commonly used as a loss function. \cite{lygerakis2024edvaeentropydecompositionelbo}. It consists of two parts, the reconstruction likelihood $\mathcal{L}_{\text{rec}}$ and the prior constraint $\mathcal{L}_{\text{reg}}$:
\begin{align}
\mathcal{L}_{\text{ELBO}}(\theta, \phi; x) &= \mathcal{L}_{\text{rec}}(\theta, \phi; x) + \mathcal{L}_{\text{reg}}(\phi; x)  
\end{align}
where:
\begin{align}
\mathcal{L}_{\text{rec}}(\theta, \phi; x) &= \mathbb{E}_{q_\phi(z|x)}[\log p_\theta(x|z)] \\
\mathcal{L}_{\text{reg}}(\phi; x) &= -D_{\text{KL}}(q_\phi(z|x) \| p(z))
\end{align}
$\mathcal{L}_{\text{reg}}$ is the negative \acrfull{kld} \cite{10.1214/aoms/1177729694}, which can be further formulated as:
\begin{equation}
    D_{\text{KL}}(q_\phi(z|x) \| p(z)) = \mathbb{E}_{q_\phi(z|x)}\left[\log \frac{q_\phi(z|x)}{p(z)}\right]
\end{equation}
$D_{\text{KL}}$ is always non-negative $(\geq 0)$ and is a statistical method used to quantify the proximity between two probability distributions \cite{shlens2014notes}. 
The reconstruction likelihood $\mathcal{L}_{\text{rec}}$ can be computed in different ways depending on the nature of the input data. For binary data, it is typically computed as a binary cross-entropy loss:
\begin{equation}
\mathcal{L_\text{rec}}(\theta, \phi; x) = - \mathbb{E}_{q_\phi(z|x)}[\mathcal{L}_{\text{BCE}}(x, z)]
\end{equation}
where $\mathcal{L}_{\text{BCE}}(x, z)$ is defined as:
\begin{equation}
\mathcal{L}_{\text{BCE}}(x, z) = \frac{1}{N} \sum_{i=1}^{N} \left( x_i \log p_\theta(x_i|z) + (1 - x_i) \log(1 - p_\theta(x_i|z)) \right)
\end{equation}
For continuous data, more novel adaptations \cite{lygerakis2024edvaeentropydecompositionelbo} often use the \acrshort{mse} loss as shown in equation \ref{eq:mse}, thus giving:
\begin{equation}
    \mathcal{L}_{\text{rec}}(\theta, \phi; x) = \mathbb{E}_{q_\phi(z|x)}[\|x - \hat{x}\|^2]
\end{equation}
where $\hat{x} = p_\theta(x|z)$ is the reconstructed input.
The two losses combined aims at providing a total loss that balances reconstruction quality and the prior regularization \cite{lin2019balancingreconstructionqualityregularisation}.

\subsubsection{Minimizing the ELBO Loss}
The objective in training a \acrshort{vae} is to minimize the negative \acrshort{elbo}, which is equivalent to maximizing the \acrshort{elbo} itself. This optimization problem can be formulated as:

\begin{equation}
    \theta^*, \phi^* = \argmin_{\theta, \phi} \mathbb{E}_{x \sim p_{\text{data}}(x)}[-\mathcal{L}_{\text{ELBO}}(\theta, \phi; x)]
\end{equation}

where $\theta^*$ and $\phi^*$ are the optimal parameters for the decoder and encoder, respectively. By minimizing the negative \acrshort{elbo}, we simultaneously optimize for better reconstruction of the input data (through $\mathcal{L}_{\text{rec}}$) and a latent space distribution that closely matches the prior (through $\mathcal{L}_{\text{reg}}$). This process encourages the \acrshort{vae} to learn a meaningful and structured latent representation of the input data while maintaining the ability to generate new samples \cite{kingma2022autoencodingvariationalbayes}.
\subsection{Data Processing and Mixed Precision Training}
\label{back:data}

\subsubsection{Data Normalization}

Normalization is a technique by which data is transformed from its original scale to a more standard scale \cite{ali2014data}. These techniques are normally used when the dataset has elements of different ranges. Normalization can contribute to faster convergence, and it's why they are as commonly used when preprocessing data. 

\textbf{MinMax Normalization} 

This algorithm transforms data to a specified range, most often $[0, 1]$, but it can also be $[-1, 1]$ or any other range.
The minmax normalization function can be formulated as follows:
\begin{equation}
   x_{\text{normalized}} = \dfrac{x - x_{min}}{x_{max}-x_{min}}
\end{equation}
\vspace{0.2cm}
where $x$ denotes the data, $x_{min}$ and $x_{max}$ is the minimum and maximum in $x$. \\
%
\subsubsection{Half-Precision and Mix-Precision Training}

Working with large datasets and \acrshort{dnn}s can be rather time- and resource-consumptive. One can cast the datatype from single precision to half-precision, along with weights, biases, and losses to address this. This will reduce the loss of accuracy and information, but it can drastically lower memory consumption and decrease training time. It is important to note that datacasting occurs with normalization techniques; the order in which the operation happens first is quintessential. If the data is cast to half precision before normalization, the normalization will be based on a slightly inaccurate representation of the data, but the computation of the normalized data will be faster. If normalization were to occur first, less detail about the data would be lost, but at the expense of being more computational intensive.\\

\begin{figure}[!h]
    \centering
    \includegraphics[width=0.8\linewidth]{figures/floats.png}
    \caption{Float16 have a limited amount of bits to represent floating point numbers compared to Float32}
    \label{fig:floats}
\end{figure}

To combat vanishing- or even exploding gradients, we can instead do some operation in \texttt{Float32} while others in \texttt{Float16}. Mixed precision training, introduced in 2018 \cite{micikevicius2018mixed}, is a technique where a neural network's weights, activations, and biases are stored in single precision while the data stays in its original format. It allows for reduced memory consumption while also speeding up the operations of deep neural nets. Additionally, the amount of $\si{\kilo\watt\hour}$ required to train the neural nets would decrease, thus reducing both the cost and the environmental tax of training neural nets. This becomes more important with larger datasets, models, and the sheer amount of GPUS required to train massive workloads. This introduces the concept of loss scaling, where the losses must be adjusted based on the weights. \\

\mycomment{
\subsection{Dataloaders}

If the datasets contain information about the data and how to retrieve a single instance, the dataloaders job is to create an object that can be iterated over, containing $n$ amount of batches, and transferring these data to the wanted devices. In the case of data parallel multi-gpu training, when the data is loaded, it's \textit{sharded} across the different gpus, in a manner that balances the load of each gpu. Let's say we have a batch  of size $[4, 5, 5]$ and we have two gpus available. The dataloader can split this Tensor in two batches, where each of the gpus get a tensor of size $[2, 5, 5]$. By sharding the data along the first axis, ideally we can half the amount it takes, not taking data transfer time into consideration. 

\begin{figure}[!h]
    \centering
    \includegraphics[scale=0.4]{figures/sharding.png}
    \caption{Example of data sharding with 2 gpus, and a original Tensor of size [4,5,5]}
    \label{fig:sharding}
\end{figure}


\textbf{Parallel loading}

When iterating over a dataloader, each element of the batch is retrieved and undergoes transformations. Depending on the batch size and  the size of the data, this procedure can be very resource-intensive and time-consuming. To mitigate this, we can introduce the concept of parallel batch loading. Instead of gathering and transforming the data sequentially, one can leverage available resources to do this operation in a parallel manner. Algorithm \ref{alg:parallel-batch-loading} details how such an operation is conducted.


\begin{algorithm}
\caption{Parallel Batch Data Loading}\label{alg:parallel-batch-loading}
\begin{algorithmic}
\Require BatchIndices, N
\Ensure BatchData, LoadSingleData
\State Initialize empty list BatchData
\State Create ThreadPool with N threads
\For{each Index in BatchIndices}
    \State Submit LoadSingleData(Index) to executor
\EndFor
\For{each completed Future from executor}
    \State Data $\gets$ Future.result()
    \State Append Data to BatchData
\EndFor
\State \Return BatchData
\end{algorithmic}
\end{algorithm}
}

\subsubsection{Overfitting and Early Stopping}

Overfitting is a common problem within \acrlong{ml} \cite{srivastava2014dropout}. It occurs when a model is trained to close to the training data, and fails to capture features of other data.In addition to popular techniques such as dropout \cite{srivastava2014dropout}, \textit{early stopping} is a regularization technique that aims to avoid overfitting. When training an optimizer such as \acrshort{adam} \cite{kingma2017adam}, or \acrshort{sgd} , one can notice when a model is overfitting by studying the validation loss. If the validation loss $L_v$ starts increasing, the early stop mechanism will stop the training altogether if no improvement of $L_v$ by at least $\epsilon$ is found after $p$ number of epochs. This can described accordingly; we stop at epoch $T$ if:
\begin{equation}
\forall i \in \{T-p+1, ..., T\}: L_v(i) > L_v^* - \epsilon
\end{equation}
where $L_v^*$ is defined as:
\begin{equation}
L_v^* = \min_{j=1}^{T} L_v(j)
\end{equation}

\subsubsection{Parallelism within \acrlong{dl}}

With rapidly evolving deep learning architectures and workloads, the importance of scalable training grows each year. It is estimated that these networks grow $\qty{1.5}{x}$ each year \cite{9499913}, making parallelization a vital topic in \acrshort{ai} to accommodate ever-increasing memory needs. Several different hardware accelerators have been created to best accommodate these needs; the most apparent of these are \acrshort{gpu}s. By workers, we mainly refer to \acrshort{gpu}s, but this could also be other types of hardware accelerators such as TPUs \cite{jouppi2023tpu}.



\subsubsection{Data Parallelism}

Data parallelism refers to partitioning data across multiple workers. Given a dataset $X$, we can split $X$ across the workers and store a copy of the model $M$ on each worker, calculate gradients across them all and update the trainable parameters for $M$ Example of how data can. 

\begin{figure}[!h]
    \centering
    \includegraphics[width=0.7\linewidth]{figures/sharding.png}
    \caption{Example of data parallel training, where a tensor $M$ of dimensions $[4,5,5]$ is split on the first axis across two \acrshort{gpu}s, leading to each \acrshort{gpu} receiving a tensor of dimension $[2,5,5$.}
    \label{fig:dataparallel}
\end{figure}

