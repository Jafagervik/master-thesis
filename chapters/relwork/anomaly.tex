\section{Anomaly Detection}
\label{relwork:anomaly}


Anomaly detection, sometimes referred to as outlier detection, is highly relevant within \acrshort{das} research. In 2017, several classical \acrshort{ml} techniques such as Gaussian Mixture Model (GMM), Hidden Markov Model (HMM), Naive Bayes (NB), and Restricted Boltzmann Machine (RBM) are being compared to discriminative models including \acrshort{ann}s \cite{app7080841}. Variations of isolation forests is shown to be able to perform fault detection for mining conveyors\cite{WIJAYA2022110330}. \\

As previously mentioned in chapter \ref{chap:introduction}, label-free anomaly detection has the advantage of requiring a lot less manual labour and can be adapted to multiple datasets. A model that require only normal-state data, utilizing both autoencoders in combination with the K-means clustering technique, have proven to yield great results, even beating supervised methods \cite{s23084094}. \\ 

a, \cite{10.14778/3538598.3538602} \cite{10.1145/3444690}.

These papers show how anomaly detection on \acrshort{das} data is highly relevant, and how . However, they do not necessarily concern themselves with available computational power, overall memory consumption, or how to optimize these algorithms for real-time environments, where not only accuracy and fault tolerance, but also inference speed is of utmost importance.