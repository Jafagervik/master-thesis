\section{Artificial Intelligence Framework}

\subsection{Julia - High Performance Data Driven Scientific Computing}

Created in 2009, first released in 2012 as part of a master thesis \cite{juliaMs} and further a doctural thesis \cite{juliaPHD}, Julia is a high performance, dynamically typed programming language with the goal of being a fast and performant language like C, the simplicity of Python, linear algebra extensions like matlab, and statistics like R \cite{julia}.  \\ 

Like Python, Julia can be used in a script-like fashion through its \acrfull{repl}. Notebook style programming has been a standard way for data analysts to write their code, and Jupyter notebooks support Julia as well through the \texttt{IJulia.jl} package. 

Julias fluent type system, accompanied by easy syntax, high performance, \acrshort{repl} tools makes it a great contender for data analysis. We've previously proven how Julia effectively deals with I/O operations, \cite{projthesis}.


DOI \cite{doi:10.1137/141000671}

\subsection{Flux.jl}
\label{back:flux}

Flux is a machine learning library written entirely in Julia released and published in 2018 \cite{Flux.jl-2018, Innes2018}. It allows the user to write their own machine learning libraries. \acrshort{gpu} support is also native, through the inclusion of \texttt{CUDA.jl} \cite{Besard_2019}. We will be writing all of our models using Flux, or \texttt{Zygote.jl}, which \texttt{Flux.jl} is based on. 

Although \texttt{Flux.jl} is the preferred way to work with \acrshort{ai} in Julia, other prominent alternatives exists as well. \texttt{MLJ.jl} \cite{blaom2020flexible} \cite{Blaom2020} is a framework provided by the Alan Turing Institute\texttrademark, providing interfaces and functions for working with about 200 machine learning models. 

\texttt{Tensorflow.jl} is a Julia package which wraps Tensorflow functions from Python to be able to work with ethem

Julia has support for working on outlier detection as follows: \cite{muhr2022outlierdetectionjl}.

A list of all packages used in addition to Flux.jl can be found in the appendix \ref{app:packages}.
