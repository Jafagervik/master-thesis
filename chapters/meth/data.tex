\section{Data}

Talk about the input


\begin{table}[h]
\begin{tabular}{|r|r|r|r|r|}
\hline
\textbf{1}                & \textbf{2}               & \multicolumn{1}{c|}{\textbf{...}} & \textbf{n-1}             & \textbf{n}               \\ \hline
1f-3                      & 2.3f-5                   & 4f-4                              & 3.4f-6                   & 3f-1                     \\ \hline
3f-1                      & 3f-1                     & 3f-1                              & 3f-1                     & 3f-1                     \\ \hline
\multicolumn{1}{|c|}{...} & \multicolumn{1}{c|}{...} & \multicolumn{1}{c|}{...}          & \multicolumn{1}{c|}{...} & \multicolumn{1}{c|}{...} \\ \hline
4f-2                      & 3f-1                     & 3f-1                              & 3f-1                     & 3f-1                     \\ \hline
\end{tabular}
\caption{Table explaining how the table looks like}
\label{fig:datatable}
\end{table}


As opposed to previous versions that stored the data in \texttt{Pandas} dataframe where 


TODO: Figure over dataflow  

\subsection{Pre processing}

After the data has been loaded \cite{projthesis}, it's still not ready for processing. 

The function \lstinline|process_DAS_data| takes in the signal data, perform a tukey \ref{dsp:tukey} window algorithm over it. We then run a bandpass filter algorithm with Butterworth over it to extract the and 

\begin{figure}
    \centering
    \lstinputlisting{code/procdasdata.jl}
    \caption{Process DAS data function}
    \label{fig:procdasdata}
\end{figure}


After applying a window function and a filter function to combat oscilleration, we're now ready to run an fft over this data. 


The 

\begin{figure}[h]
    \centering
    \includegraphics{figures/banenor_20210831.png}
    \caption{Heatmap of the processed train data during the 31st of august 2021 from Trondheim to Storen}
    \label{fig:trdstrdas}
\end{figure}

\subsection{Data preparation before } 

Now that the data has been successfully pre-processed, we still need to 



\subsubsection{Data augmentation}
