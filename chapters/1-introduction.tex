\chapter{Introduction}
\label{chap:introduction}

In this very first chapter, we cover our motivation for this projects, our goals and research questions, what contributions this paper have given as well as an outline for the thesis.


Digital signal processing has over the last 100 years gone from almost non-existent to utmost critical importance. There are tons of different devices for recording signals from all kinds of environments. From more well known ones such as microphones and recorders, to less commercial technologies. \acrfull{das} is a rather new technology that allows for real-time analysis over fiber-optical cables, and gives us highly sensitive senor data to work with. This technology has gained more recognition within the last decade, and have real world applications spanning from earthquake and whale detection, to landslide, CO2 and ground water level analysis \\

\begin{figure}[!h]
    \centering
    \includegraphics[width=0.7\linewidth]{figures/das.png}
    \caption{Showcase of how \acrshort{das} signals are recorded}
    \label{fig:das-fig}
\end{figure}



\section{Motivation and Challenges}

\subsection{Context} 

\acrfull{das} is a rather new technology that allows for real-time analysis over fiber-optical cables. This technology has gained more recognition within the last decade, and due to their high sensitivity, \acrshort{das} systems can detect subtle environmental changes and anomalies. Analyzing these irregularities is a common and crucial task in various fields, and it can be applied to tasks spanning landslide and earthquake detection as well as railroad and maritime monitoring. The ability to process and interpret \acrshort{das} data effectively is essential for extracting meaningful insights from these complex measurements. \\

\begin{figure}[!h]
    \centering
    \includegraphics[width=0.7\linewidth]{figures/das.png}
    \caption{Showcase of how \acrshort{das} signals are recorded}
    \label{fig:das-fig}
\end{figure}

Traditionally, clustering-based \acrfull{ml} techniques such as K-MEANS \cite{hartigan1979k} DBSCAN \cite{ester1996density} have been quite popular for anomaly detection \cite{anomaly}. Across the last years, a popular modification to the DBSCAN algorithm, HDBSCAN \cite{rahman2016hdbscandensitybasedclustering}, has also shown prowess in clustering-based anomaly detection \cite{ariyaluran2022clustering}. However, these methods often require manual feature engineering, require labeled datasets, or generally do not scale to large datasets. 

\begin{figure}[!h]
    \centering
    \includegraphics[scale=0.4]{figures/anolay_line.png}
    \caption{Example of anomalies in a time series}
    \label{fig:anomaly_example}
\end{figure}

\acrshort{das} technology in itself has now started garnering attention for research, and several papers have previously studied how one can process this data. \acrshort{ai} and \acrshort{ml} models have been constructed for looking at time series data and analyzing sensor data, although several of these have been studied.  Only recently has \acrshort{ai}


\subsection{Motivating challenges}. 

\acrfull{cgf} spend a lot of time and resources on processing and analyzing \acrshort{das} data. Current tools for processing are quite slow and do not utilize parallelization techniques that have the potential to drastically speed up computations. Additionally, analysis often uses more traditional signal processing techniques, not leveraging the potential benefits of more novel \acrshort{ann} methods. \\ 


Recorded \acrshort{das} data has the potential to become quite large, up to several terabytes per experiment, underlying the importance of efficient algorithm design and processing techniques. Generally, languages such as Python and Matlab are used for DAS analysis due to their framework for data science applications. However, these programming languages are not designed for data-intensive applications without having to leverage languages such as C. Julia is a more novel language aimed at both data science and in general \acrfull{hpc} applications, and could prove really powerful as an alternative to an existent program.


However, with the upcoming of \acrshort{das}, both unsupervised and supervised \acrfull{dl} methods have proven to produce even better results for anomaly detection. For \acrshort{das} data specifically, both scalability and manual labeling can become quite tedious or outright non-feasible. For 

In later years, unsupervised learning has returned after the explosion of generative models [CITE]. Compared to their supervised alternatives, unsupervised do not require manual labeling. They're, therefore, not prone to some of the more common problems within supervised methods, such as detecting irregular events.   are well suited for detecting novel anomalies \cite{wei2022lstmautoencoder, srivastava2016unsupervised} compared to its supervised alternatives, and do not require manual labeling. This makes them 

Current autoencoder-based approaches to anomaly detection of \acrshort{das} do not emphasize the overall memory consumption or the conversion of models to a real-time environment. This 



Previous work on this data \cite{projthesis} revolved around processing \acrshort{hdf5} files as fast and efficiently as possible, trying to parallelize already existent code, and take advantage of newer technologies, such as Julia.

\section{Goals}

Our goals for this thesis are as following: 

\begin{enumerate}
    \item Find out if Julia is a well suited  language when it comes to big data and \acrshort{ai}.
    \item What kinds of unsupervised models can work well with \acrshort{das} data.
    \item If our tool can be efficiently used by other members at \acrshort{cgf}.
\end{enumerate}


\subsection{Research Questions}

In addition to our goals, the following are a set of questions we want answers to by the end of the article

\section{Contributions}

(Note to Ole)
Goal: 
    1. Develop, or improve tools that can process \acrshort{das} data and detect anomalies, both opensource and for CGF
    
RQ:
    1. Is Float16 training sufficient for training \acrshort{das} data in the context of data reconstruction and anomaly detection
    2. How does the different autoencoders compare, and can we make a convolutional variational autoencoder for das data

In this thesis, we study processing and autoencoder-based anomaly detection within \acrshort{das} data. Our work has led to several contributions, including:

\begin{itemize}
    \item \textbf{CVAE}: We present a \textit{novel} \acrfull{cvae} model for anomaly detection on \acrshort{das} data.
    \item \textbf{Autoencoder-based anomaly detection}: We compare the effectiveness of different autoencoders for anomaly detection on dense \acrshort{das} data. In particular, we explore anomaly detection on \acrshort{das} data as an image reconstruction problem, contrary to a time-series problem. Additionally, we discuss the effectiveness of half-precision training and inference.
    \item \textbf{Julia for datascience and \acrshort{ai}}: We evaluate Julia as a programming language for developing highly performant programs and \acrshort{ai} programming.
    \item \textbf{Software}: The following software has been produced as a part of this thesis:
    \begin{itemize}
        \item \textbf{Judas}: A software package developed in Julia for processing \acrshort{das} data. Initially introduced in our project thesis \cite{projthesis}, Judas is now fully operational but only available for members of \acrshort{cgf}. 
        \item \textbf{TinyDAS}: An open-source program written in Python, specifically designed for training and evaluation of autoencoder models, as well as performing anomaly detection on \acrshort{das} data \footnote{\url{github.com/Jafagervik/TinyDAS}}. This program contains code and hyperparameters for 4 different autoencoders.  We establish how Tinygrad \cite{tinygrad} as a software package can be used to create hardware agnostic \acrshort{ai} programs that are scalable across multiple accelerators without changing source code. 
        \item \textbf{JudasNET}: An open-source repository with examples of autoencoders written in Julia \footnote{\url{github.com/Jafagervik/JudasNET}}.
    \end{itemize}
\end{itemize}

Overall, we seek to improve \acrshort{das} data processing and compare the effectiveness of autoencoder models for anomaly detection on this data. In particular, we hope that members of \acrshort{cgf} can use and improve these tools to further \acrshort{das} research.
\section{Thesis outline}

The following list is an outline over the rest of the thesis, and what will be presented for each section. \\

\textbf{Chapter 1: Introduction} - We present the problems, what we want to find out and our motivation for this project. \\

\textbf{Chapter 2: Background and Theory} - We go more in dept about theory regarding both relevant \acrshort{dl} architectures and their applications to our problems, as well as some introduction to applicable signal processing techniques.  \\

\textbf{Chapter 3: Literature Review} - Here we discuss relevant literature, both on \acrshort{ai} for signal processing in general, but also around \acrshort{das} data. \\

\textbf{Chapter 4: Method} - We cover all our practical work, and implementation decisions. This includes data processing, \acrshort{api} design, network architecture and experiments. \\

\textbf{Chapter 5: Results and Discussions} - We present our findings, compare results, discuss different outcomes and about Julia in general. \\

\textbf{Chapter 6: Conclusion and Further Work} - This final chapter concludes our findings. We answer the questions asked in \textbf{Chapter 1}, and try to see where all this leaves us going forward. \\

