\chapter*{Abstract}

\acrfull{das} has become a prominent technology in signal processing over the last decade. Current anomaly detection methods for \acrshort{das} data often prioritize accuracy over efficient preprocessing and resource optimization. This thesis examines parallel processing techniques for dense-sampled, large-scale \acrshort{das} data and the application of smaller autoencoders for fast anomaly detection.

We present two programs: 1) \texttt{Judas.jl}, a Julia package for loading and processing \acrshort{das} data, and 2) \texttt{TinyDAS}, a \textit{novel} program for dataparallel training of autoencoders and efficient anomaly detection. These tools aim to automate anomaly detection in offline and online environments, potentially reducing manual intervention while maintaining high accuracy. Our methodology involves validating these programs on proprietary and open-source real-world \acrshort{das} datasets.

This research contributes to \acrshort{das} data analysis by providing robust tools for handling large-scale data and comparing different autoencoders for anomaly detection, as well as the spatial context. The findings have implications for various industries utilizing \acrshort{das} technology, including \acrfull{cgf}, offering potential enhancements in data processing pipelines and bringing awareness towards efficient designs of autoencoders. Parts of this thesis are based on my submitted project assignment in TDT4501 titled "Parallel \acrshort{das} Processing: Julia is all you need".