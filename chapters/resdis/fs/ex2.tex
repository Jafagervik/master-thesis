\subsection{Experiment \rnum{2}: Anomaly Detection}

Table \ref{tab:confusion-matrix-results} presents a comparative analysis of all four models and data types. Except for the $\beta$-VAE model, all scores are equal between all metrics. The variational models have the highest values for the positives and the lowest for the negatives. The CAE model has a higher rate of falsely predicted values yet is still close to the variational ones. Note that the AE scores in subsequent tables will be lower due to the fact that no true anomalies have been predicted.

\begin{table}[!htbp]
\centering
\begin{tabular}{l cccccccc}
\toprule
\multirow{2}{*}{\textbf{Model}} & \multicolumn{2}{c}{\textbf{TP}} & \multicolumn{2}{c}{\textbf{FP}} & \multicolumn{2}{c}{\textbf{TN}} & \multicolumn{2}{c}{\textbf{FN}} \\
\cmidrule(lr){2-3} \cmidrule(lr){4-5} \cmidrule(lr){6-7} \cmidrule(lr){8-9}
& \textbf{F32} & \textbf{F16} & \textbf{F32} & \textbf{F16} & \textbf{F32} & \textbf{F16} & \textbf{F32} & \textbf{F16} \\
\midrule
\rowcolor{gray!10} AE   & 0 & 0 & 42 & 42 & 418 & 418 & 140 & 140 \\
$\beta$-VAE  & 109 & 105 & 13 & 7 & 447 & 453 & 31 & 35 \\
\rowcolor{gray!10} CAE  & 96 & 96 & 10 & 10 & 450 & 450 & 44 & 44 \\
$\beta$-CVAE & 106 & 106 & 9 & 9 & 451 & 451 & 34 & 34 \\
\bottomrule
\end{tabular}
\caption{Confusion Matrix Values for all four models and datatypes (Float32 and Float16)}
\label{tab:confusion-matrix-results}
\end{table}

Table \ref{tab:performance-metrics-float32} highlights the key metrics when we use Float32 for anomaly detection. Except for the AE model, all models score above 90\% on accuracy, prediction, and F1-score with a low FPR.

\begin{table}[!htbp]
\centering
\begin{tabular}{l
    S[table-format=1.3]
    S[table-format=1.3]
    S[table-format=1.3]
    S[table-format=1.3]
    S[table-format=1.3]
    S[table-format=1.3]
    S[table-format=1.3]
}
\toprule
\textbf{Model} & {\textbf{Accuracy}} & {\textbf{Precision}} & {\textbf{Recall}} & {\textbf{F1-Score}} & {\textbf{FPR}} \\
\midrule
\rowcolor{gray!10} AE   & 0.697 & 0.000 & 0.000 & 0.000 & 0.091 \\
$\beta$-VAE  & 0.927 & 0.893 & 0.779 & 0.832 & 0.028\\
\rowcolor{gray!10} CAE  & 0.910 & 0.906 & 0.686 & 0.780 & 0.022  \\
$\beta$-CVAE & 0.928 & 0.922 & 0.757 & 0.831 & 0.020  \\
\bottomrule
\end{tabular}
\caption{Anomaly Detection Performance Metrics Comparison (Float32)}
\label{tab:performance-metrics-float32}
\end{table}

Like Table \ref{tab:performance-metrics-float32}, Table \ref{tab:performance-metrics-float16} displays key anomaly metrics. We only notice a small change between the results from single-precision, contrary to that of half-precision in the $\beta$-VAE model. 
\begin{table}[!htbp]
\centering
\begin{tabular}{l
    S[table-format=1.3]
    S[table-format=1.3]
    S[table-format=1.3]
    S[table-format=1.3]
    S[table-format=1.3]
    S[table-format=1.3]
    S[table-format=1.3]
}
\toprule
\textbf{Model} & {\textbf{Accuracy}} & {\textbf{Precision}} & {\textbf{Recall}} & {\textbf{F1-Score}} & {\textbf{FPR}} & {\textbf{ROC AUC}} & {\textbf{PR AUC}} \\
\midrule
\rowcolor{gray!10} AE   & 0.697 & 0.000 & 0.000 & 0.000 & 0.091 & 0.293 & 0.233 \\
$\beta$-VAE  & \textbf{0.930} & \textbf{0.938} & 0.750 & \textbf{0.833} & \textbf{0.015} & \textbf{0.969} & 0.910 \\
\rowcolor{gray!10} CAE  & 0.910 & 0.906 & 0.686 & 0.780 & 0.022 & 0.820 & 0.690 \\
$\beta$-CVAE & 0.928 & 0.922 & \textbf{0.757} & 0.831 & 0.020 & 0.967 & \textbf{0.900} \\
\bottomrule
\end{tabular}
\caption{Anomaly Detection Performance Metrics Comparison (Float16)}
\label{tab:performance-metrics-float16}
\end{table}
\subsubsection{Combined ROC- and PR-Curve using half-precision}
\begin{figure}[!htbp]
    \centering
    \begin{subfigure}[b]{\textwidth}
        \centering
        \includegraphics[width=0.7\textwidth]{figures/anomalies/combined_roc_curve.png}
        \caption{Combined ROC Curve for all models using half-precision. AE is Blue, $\beta$-VAE is yellow, CAE is green $\beta$-CVAE is red. The AE model has an upward curve, while the other models perform better, with the variational models yielding high AUC.}
        \label{fig:roccurve}
    \end{subfigure}
    \vspace{1em}
    \begin{subfigure}[b]{\textwidth}
        \centering
        \includegraphics[width=0.7\textwidth]{figures/anomalies/combined_pr_curve.png}
        \caption{Combined PR Curve for all models using half-precision. AE is Blue, $\beta$-VAE is yellow, CAE is green, and $\beta$-CVAE is red. Similar to the ROC curve, The AE model yields the lowest AUC. The variational models scores higher overall, while the \acrshort{cae} model scores in between.}
        \label{fig:prcurve}
    \end{subfigure}
    \label{fig:combined_curves}
\end{figure}
\clearpage

\subsubsection{Threshold Analysis using half-precision inference}

\begin{figure}[!h]
  \centering
  \includegraphics[scale=0.4]{figures/anomalies/ae/threshold_True.png}
  \caption{Threshhold analysis for the AE model. All graphs follow an L-shape structure, with no real peaks for the F1-score. The model still finds a threshold.}
  \label{fig:threshold_ae}
\end{figure}

\begin{figure}[!h]
  \centering
  \includegraphics[scale=0.4]{figures/anomalies/vae/threshold_True.png}
  \caption{Threshhold analysis for the $\beta$-VAE model. The threshold found for the $\beta$-VAE models is close to that of AE, but with a distinctive peak in the F1-score.}
  \label{fig:threshold_vae}
\end{figure}

\begin{figure}[!h]
  \centering
  \includegraphics[scale=0.4]{figures/anomalies/cae/threshold_True.png}
  \caption{Threshhold analysis for the CAE model. Its peak is on a different scale compared to the previous models, but the graphs follow the same outline as the one observed in Figure \ref{fig:threshold_vae}.}
  \label{fig:threshold_cae}
\end{figure}

\begin{figure}[!h]
  \centering
  \includegraphics[scale=0.4]{figures/anomalies/cvae/threshold_True.png}
  \caption{Threshhold analysis for the $\beta$-CVAE model. The graphs and best threshold found are close or equal to the ones observed in Figure \ref{fig:threshold_vae}.}
  \label{fig:threshold_vae}
\end{figure}