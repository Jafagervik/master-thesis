\chapter{Method}
\label{chap:method}

% Data
\section{Data}

\subsection{Dataset}

For this project, we will be working with data recorded from BANE-NOR, where a train is travelling between Trondheim and Storen the 31st of august 2021 \footnote{Working with national infrastructure requires security clearance, see \ref{app:conf} for more details}. Data is split between 5000 sensor channels, lasting us a day.
The data is recorded in \acrshort{hdf5} files, with each file containing 20000 samples for 10 seconds, giving us a sample rate of $2000Hz$. The


\begin{table}[h]
\centering
\begin{tabular}{|r|r|r|r|r|}
\hline
\textbf{1}                & \textbf{2}               & \multicolumn{1}{c|}{\textbf{...}} & \textbf{n-1}             & \textbf{n}               \\ \hline
1f-3                      & 2.3f-5                   & 4f-4                              & 3.4f-6                   & 3f-1                     \\ \hline
3f-1                      & 3f-1                     & 3f-1                              & 3f-1                     & 3f-1                     \\ \hline
\multicolumn{1}{|c|}{...} & \multicolumn{1}{c|}{...} & \multicolumn{1}{c|}{...}          & \multicolumn{1}{c|}{...} & \multicolumn{1}{c|}{...} \\ \hline
4f-2                      & 3f-1                     & 3f-1                              & 3f-1                     & 3f-1                     \\ \hline
\end{tabular}
\caption{Table explaining how the table looks like}
\label{fig:datatable}
\end{table}



\begin{figure}[h]
    \centering
    \includegraphics{figures/dataflow.png}
    \caption{Dataflow from we read HDF5 files to we are ready to train}
    \label{fig:dataflow}
\end{figure}

\subsection{Pre processing}

\subsubsection{Resampling and Channel decimation}

Following our previous work \cite{projthesis}, the following improvements have been made to \texttt{Judas}: 

\begin{enumerate}
    \item Judas now make use of multiple processors contra multithreading, seeing major speedups.
    \item Function \texttt{load\_DAS\_files} now writes processed data to a single file back.
    \item Methods for resampling and chann
\end{enumerate}


\subsubsection{Window and Filter operations}

After the data has been loaded \cite{projthesis}, it's still not ready for processing. 

The function \lstinline|process_DAS_data| takes in the signal data, perform a tukey \ref{dsp:tukey} window algorithm over it. We then run a bandpass filter algorithm with Butterworth over it to extract the and 

\begin{figure}[h]
    \centering
    \lstinputlisting{code/procdasdata.jl}
    \caption{Process DAS data function}
    \label{fig:procdasdata}
\end{figure}


After applying a window function and a filter function to combat oscilleration, we're now ready to run an fft over this data. 


The 

\begin{figure}[h]
    \centering
    \includegraphics{figures/banenor_20210831.png}
    \caption{Heatmap of the processed train data during the 31st of august 2021 from Trondheim to Storen}
    \label{fig:trdstrdas}
\end{figure}

\subsection{Data preparation before Training} 

After the preprocessing, some final work must be done before we can train the model. Normalization, also called data standardization, 
is a basic staple of preprocessing, we  . This normalization makes sure all the values . For our model, we use z-score normalization, 
which normalizes vales to be in the range 0 to 1. The f

\begin{equation}
    x' = \frac{x - \mu}{x}
\end{equation}

This method is implemented as such in \texttt{Flux.jl}

\begin{figure}[h]
    \centering
    \lstinputlisting{code/norm.jl}
    \caption{Flux normalization function}
    \label{fig:normalize_data}
\end{figure}

Where $\mu$ is the mean of the data, and $\sigma$ is the standard deviation of data.

\subsubsection{Data augmentation}


In this chapter, we go through our choice of language and frameworks for implementation. Furthermore, we go into detail about the datasets provided, how we improve on the previous implementation of our program and how data is processed. Finally we describe how our models are implemented, as well as a choice of parameters and architectures. \\

\section{Programming Languages and Frameworks}

Our initial decision was to continue using Julia for our \acrshort{dl} methods, and train our models on the same dataset provided by \acrshort{cgf}. We created a Julia package called JudasNET, containing code for training different \acrshort{ai} models. However, due to severe computational limitations, we were unable to continue our work on the previous decision. With only 11GB of VRAM, the single \acrshort{gpu} available quickly became a bottleneck. Not only do we need to store batches of \acrshort{das} data matrices, but we also need to store the weights and biases of our models. By switching to a open source dataset, we would be able to use \Gls{idun}, and leverage multiple \acrshort{gpu}s and in general more computational power. Since the BANENOR dataset is close sourced, we would not be able to train our models on any other resources outside of those provided by \acrshort{cgf}.

Not only were our attempts at training models unsuccessful, Julia's main framework for \acrshort{ml} training \texttt{Flux.jl} does not provide built in tools for multigpu training. \\  

The more obvious choice would now be to use the Python package \texttt{Pytorch}, which is well established, documented and supports not only data parallel training \footnote{  \href{https://pytorch.org/docs/stable/generated/torch.nn.DataParallel.html}{https://pytorch.org/docs/stable/generated/torch.nn.DataParallel.html}}, but also distributed data parallel training \footnote{\href{https://pytorch.org/tutorials/intermediate/ddp_tutorial.html}{https://pytorch.org/tutorials/intermediate/ddp\_tutorial.html}}. What needs to be mentrioned is that Pytorch is heavily optimized for CUDA and NVIDIA \acrshort{gpu}s, and in general performs significantly better on these accelerators, compared to other alternatives such as AMD, NV, METAL and so on. \\ 

We want to create models where we don't need to change much of our code to run on different accelerators. We also want \acrshort{cgf} to  quickly be able to both continue training and running their own models. NVIDIA \acrshort{gpu}s are generally expensive, many of them reaching prices of tens of thousands of dollars per \acrshort{gpu}. We realize that this would not be feasible for \acrshort{cgf} to invest in right now, and thus we decided to look for other frameworks which supports a broader range of hardware accelerators. \\ 

Due to these constraints, we believe Tinygrad \cite{tinygrad} will be a well suited framework for our usecase.

\subsection{TinyGrad}











\mycomment{



DOI \cite{doi:10.1137/141000671}

\subsection{Flux.jl}
\label{back:flux}

Flux is a machine learning library written entirely in Julia released and published in 2018 \cite{Flux.jl-2018, Innes2018}. It allows the user to write their own machine-learning libraries. \acrshort{gpu} support is also native, through the inclusion of \texttt{CUDA.jl} \cite{Besard_2019}. We will be writing all of our models using Flux, or \texttt{Zygote.jl}, which \texttt{Flux.jl} is based on. 

Although \texttt{Flux.jl} is the preferred way to work with \acrshort{ai} in Julia, other prominent alternatives exists as well. \texttt{MLJ.jl} \cite{blaom2020flexible} \cite{Blaom2020} is a framework provided by the Alan Turing Institute\texttrademark, providing interfaces and functions for working with about 200 machine learning models. 

\texttt{Tensorflow.jl} is a Julia package which wraps Tensorflow functions from Python to be able to work with ethem

Julia has support for working on outlier detection as follows: \cite{muhr2022outlierdetectionjl}.

A list of all packages used in addition to Flux.jl can be found in the appendix \ref{app:packages}.

}

We start at the same spot we left off after the project thesis. Before we go on to explain what's been done this semester, lets recap. 

We initially got started by just getting a drive with lots of \acrshort{das} data, and a script from \acrfull{asn} which set up us perfectly. We chose to rewrite python code to a language similar to that of Matlab and Python. We landed on Julia, and with its broad ecosytstem it has support for all kinds of development we'd ever need. \\

We first wrote a 1:1 copy of the script, and then parallelized parts of the code until we had a version that could handle large amounts of data in parallel. We also wrote methods for being able to run a window over and \\ 


\section{Overview}

Our product is split into two apis. Those being \texttt{Judas} and \texttt{JudasNET}. Judas is the direct continuation from \cite{projthesis}
The \acrshort{api} that's being created is called \texttt{Judas} (Julia and DAS) and is split in 3 modules as well as a seperate Utils file as shown in \ref{fig:ccuda}

\begin{figure}[h]
    \centering
    \includegraphics[scale=.6]{figures/judas_overview.png}
    \caption{Overview over our package Judas}
    \label{fig:judasoverview}
\end{figure}

% DAS
\section{Judas - Software Frameworj}

The DAS folder has not changed much substantially. The main difference from before is how we now instead of writing matrix data to one large binary file, data is split into multiple files, and only read when needed.

Contrary to what was mentioned before, the output of the function \texttt{load\_DAS\_files} has actually changed. Previously, we stored a whole vector of the timestamps for each sample. Not only was this cistly, but actaully totally redundant. If the timestamp of the first row is known, the sampling rate $T$ and which row to look at, one can instead calculate the timestamp like this: 
\lstinline|start_time + MilliSecond(idx * T * 1000)|. This inplace calculation can be done multiple times effectively in Julia using the broadcast operator (.). This ensures that we don't lose essential information before running our data through through the autoencoder

\begin{figure}[h]
\centering
\begin{subfigure}{.5\textwidth}
  \centering
  \lstinputlisting{code/dasstructold.jl}
  \caption{Old DAS Struct}
  \label{fig:olddasstc}
\end{subfigure}%
\begin{subfigure}{.5\textwidth}
  \centering
  \lstinputlisting{code/dasstruct.jl}
  \caption{New Layout for DAS struct}
  \label{fig:newdasstc}
\end{subfigure}
\caption{Comparison between different versions of the DAS struct}
\label{fig:dasstccmp}
\end{figure}




\textbf{API Usage}

\begin{figure}[h]
    \centering
    \lstinputlisting{code/apiusage.jl}
    \caption{How to use our API}
    \label{fig:apiusage}
\end{figure}



% NETWORK ARCH
\section{Network Architecture}

The model consists of

\begin{itemize}
    \item Batch size 
    \item skdfj
    \item sdfk
\end{itemize}

\begin{table}[h]
    \centering
    \begin{tabular}{ l c r }
      \hline
      Layer (type) & Output shape & Param \# \\ \hline
      lstm\_1 (LSTM) & (None, 10, 256) & 1234678 \\ \hline
      lstm\_1 (LSTM) & (None, 10, 256) & 1234678 \\ \hline
      lstm\_1 (LSTM) & (None, 10, 256) & 1234678 \\ \hline
      lstm\_1 (LSTM) & (None, 10, 256) & 1234678 \\ \hline
      lstm\_1 (LSTM) & (None, 10, 256) & 1234678 \\ \hline
      Total params: & & \\
      Trainable params: & & \\ 
      Non-Trainable params: 0 & & \\ \hline
      
    \end{tabular}
    \caption{Parameters for our architecture}
    \label{tab:archparams}
\end{table}



\section{SignalProcessing.jl}

After data has been read and processed, it's ready to be processed by signal processing functions 

\section{AI.jl}

The main aspects of this code lies within the module \texttt{AI.jl}. 


\subsection{JudasNET}

\texttt{JudasNET} is the architecture of choice for interpreting data

\begin{figure}{h}
    \centering
    \includegraphics[scale=.6]{figures/judasnet.png}
    \caption{Flowchart over JudasNET}
    \label{fig:judasnet}
\end{figure}

\subsubsection{Modification to Flux.jl}

As mentioned previously, one of the main benefits with Julia packages is the ability to easily add new functionality to packages. When working with \texttt{Flux.jl}, this has certainly been useful. For our research, we had to write our own LSTM functors to be able to achieve a LSTM Autoencoder.  \\

The first of this functors is a regular LSTM, but with \texttt{relu} as the activation function for the different gates. Secondly, we rewrote the TimeDistributed layer \cite{keras} for Flux since there is no corresponding alternative as of the writeday of this paper. The final layer we had to recreate is what's known as a \texttt{RepeatVector} in Keras. When we have all of these together, we can finally put them together and write our model. 

\begin{figure}[h]
    \centering
    \lstinputlisting[language=Julia]{code/judasnet.jl}
    \caption{JudasNET flux model}
    \label{fig:fluxjudasnet}
\end{figure}



We're using \acrfull{adam} as the choice of our optimizer

\begin{equation}
    \omega - \alpha \frac{v_{dw}}  {\sqrt{s_{dw}} + \epsilon}
\end{equation}


\input{chapters/meth/experiment}