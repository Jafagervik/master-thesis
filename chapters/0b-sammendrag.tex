\chapter*{Sammendrag}

Distribuert akustisk sensing (DAS) har blitt en fremtredende teknologi innen signalbehandling det siste tiåret. Nåværende metoder for anomali-detektering i DAS-data prioriterer ofte nøyaktighet fremfor effektiv forhåndsbehandling og ressursoptimalisering. Denne avhandlingen undersøker parallelle prosesseringsteknikker for tett-samplede, storskala DAS-data og anvendelsen av mindre autoenkodere for rask anomali-detektering.

Vi presenterer to programmer: 1) \texttt{Judas.jl}, en Julia-pakke for innlasting og prosessering av DAS-data, og 2) \texttt{TinyDAS}, et \textit{nytt} program for dataparallell trening av autoenkodere og effektiv anomali-detektering. Disse verktøyene har som mål å automatisere anomali-detektering i- og utenom sanntidsmiljøer, med potensiale for å redusere manuell intervensjon samtidig som høy nøyaktighet opprettholdes. Vår metodikk innebærer validering av disse programmene på proprietære og åpen kildekode-baserte DAS-datasett fra virkelige verden.

Denne forskningen bidrar til DAS-dataanalyse ved å tilby robuste verktøy for håndtering av storskala data og ved å sammenligne ulike autoenkodere for anomali-detektering. Funnene har implikasjoner for ulike industrier som benytter DAS-teknologi, inkludert karbonfangst og -lagring (CCS), og tilbyr potensielle forbedringer i databehandlingsrørledninger samt økt bevissthet om effektive design av autoenkodere. Deler av denne avhandlingen er basert på min innleverte prosjektoppgave i TDT4501 med tittelen "Parallel DAS Processing: Julia is all you need".

