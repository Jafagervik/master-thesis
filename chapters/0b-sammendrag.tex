\chapter*{Sammendrag}
Distribuert akustisk sensing (DAS) har blitt en utbredt teknologi det siste tiåret, men nåværende prosesserings- og anomalideteksjonsmetoder prioriterer ofte nøyaktighet fremfor effektivitet. Denne avhandlingen tar for seg utfordringen med å effektivt prosessere DAS-data samtidig som effektive anomalideteksjonsevner opprettholdes. Vi undersøker parallelle prosesseringsteknikker for storskala, tett-samplet DAS-data og utforsker anvendelsen av kompakte autoencodere for rask anomalideteksjon.

Vi presenterer to verktøy: Judas, en pakke for effektiv lasting og prosessering av DAS-data, og TinyDAS, et skalerbart rammeverk for trening av autoencodere og utførelse av anomalideteksjon. Disse verktøyene har som mål å betydelig redusere dataressurser og prosesseringstid for håndtering av DAS-data. Vi validerer disse programmene ved bruk av proprietære og åpen kildekode-datasett, med fokus på jernbaneovervåking og jordskjelvdeteksjon.

Vår forskning viser at distribuert minnekartlegging kan redusere minnekravene for DAS-dataprosessering betydelig. I motsetning til eksisterende metoder som laster hele datasett inn i minnet, bruker vår teknikk distribuert binær filsplitting og on-demand-lasting, noe som muliggjør effektiv håndtering av storskala DAS-data med minimal minneoverhead. Videre viser vi at halv-presisjons inferens på en kompakt konvolusjonell autoencoder med bare 47k parametere kan oppdage anomalier i store matriser med 91\% nøyaktighet på bare 5,6ms, samtidig som den er trent på potensielt anomale data.

Dette arbeidet fremmer mer effektive teknikker for prosessering av DAS-data, som kan være til fordel for ulike industrier som benytter DAS-teknologi, inkludert infrastrukturovervåking og geofysisk varsling. Ved å forbedre databehandlingspipelines og fremme effektive autoencoder-design, baner vårt arbeid ved NTNU Senter for Geofysisk Varsling vei for mer utbredte og ressurseffektive anvendelser av DAS-teknologi i sanntidsscenarier.
