\chapter*{Sammendrag}

Distribuert akustisk sensing (DAS) har blitt en utbredt teknologi det siste tiåret, men de nåværende metodene for prosessering og anomalideteksjon prioriterer ofte nøyaktighet fremfor effektivitet. Denne oppgaven tar for seg utfordringen med å effektivt prosessere DAS-data samtidig som man opprettholder effektive evner for anomalideteksjon. Vi undersøker parallellprosesseringsmetoder for storskala, tettprøvet DAS-data og utforsker anvendelsen av kompakte autoenkodere for rask anomalideteksjon.

Vi presenterer to verktøy: Judas, en pakke for effektiv innlasting og prosessering av DAS-data, og TinyDAS, et skalerbart rammeverk for opplæring av autoenkodere og utføring av anomalideteksjon. Disse verktøyene har som mål å redusere databehandlingsressurser og prosesseringstid betydelig for håndtering av DAS-data. Vi validerer disse programmene ved å bruke både proprietære og åpne DAS-datasett, med fokus på overvåking av jernbane og jordskjelvdeteksjon.

Vår forskning viser at minnekartlegging kan redusere minnekravene for DAS-databehandling betraktelig. I motsetning til eksisterende metoder som laster hele datasett inn i minnet, benytter vår teknikk distribuert binær filsplitting og behovsstyrt innlasting, noe som muliggjør effektiv håndtering av storskala DAS-data med minimal minnebruk. Videre viser vi at inferens i halv presisjon på en kompakt konvolusjonsbasert autoenkoder med bare 46k parametere kan oppdage anomalier i store matriser med 86\% nøyaktighet på bare 5,5 ms.

Dette arbeidet fremmer mer effektive teknikker for behandling av DAS-data, noe som kan være til nytte for ulike bransjer som benytter DAS-teknologi, inkludert infrastrukturovervåking og geofysisk prognosering. Ved å forbedre databehandlingsmetoder og fremme effektivt design av autoenkodere, baner vårt arbeid ved NTNU Senter for Geofysisk Prognosering vei for mer utbredte og ressursvennlige anvendelser av DAS-teknologi i sanntidsscenarioer.

