\chapter*{Sammendrag}

Denne avhandlingen undersøker parallell prosessering av distribuert akustisk sensing (DAS)-data og anvendelsen av autoencodere for anomalideteksjon på disse dataene. Forskningen adresserer det økende behovet for effektiv prosessering og analyse av storskala \acrshort{das} data, med potensielle anvendelser som spenner fra jordskjelvdeteksjon til overvåking av jordskred. \\

Vi presenterer to nye programmer: 1) Judas.jl, en Julia-pakke for forbehandling av store volumer av \acrshort{das} data, og 2) TinyDAS, et Python-program for trening av flere autoencodere på tvers av flere GPU-er, og utføring av anomalideteksjon på DAS-data. Disse verktøyene har som mål å automatisere deteksjonen av anomalier i sanntids datastrømmer, og potensielt redusere behovet for manuell intervensjon og forbedre nøyaktigheten av lignende dataanalyse. \\

Vår metodikk involverer validering av disse programmene på både proprietære og åpen kildekode DAS-datasett fra den virkelige verden. Resultatene demonstrerer skalerbare løsninger for både dataprosessering og anomalideteksjon, og viser betydelige forbedringer i effektivitet og nøyaktighet sammenlignet med eksisterende metoder ved NTNU Senter for Geofysisk Prediksjon (CGF). \\

Denne forskningen bidrar til feltet DAS-dataanalyse ved å tilby robuste verktøy for håndtering av storskala DAS-data og utnyttelse av ulike effektive autoencodere for anomalideteksjon. Funnene har implikasjoner for ulike industrier som benytter DAS-teknologi, inkludert NTNU CGF, og tilbyr potensielle forbedringer i dataprosesseringspipelines og anomalideteksjonskapabiliteter. \\

Deler av denne avhandlingen er hentet fra eller basert på min innleverte prosjektoppgave i faget TDT4501 med tittelen "Parallel DAS Processing: Julia is all you need".

