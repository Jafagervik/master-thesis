\section{Thesis outline}

The following list is an outline over the rest of the thesis, and what will be presented for each chapter. \\

%\textbf{Chapter 1: Introduction} - We present the problems, what we want to find out and our motivation for this project. \\

\textbf{Chapter 2: Background and Related Work} - We cover theory regarding \acrshort{das} processing techniques, both relevant \acrshort{dl} architectures and their applications to our problems, as well as some introduction to applicable signal processing techniques. Finally, we discuss relevant literature and work.  \\

\textbf{Chapter 3: Method} - We cover all our practical work and implementation decisions of both programs, including program design, data processing methods, and network architectures. \\

\textbf{Chapter 4: Experiments} - We present our experiments, covering datasets, the experiments scopes, as well as evaluation metrics and experiment setup. \\

\textbf{Chapter 5: Experimental Results} - We present our results from the experiments conducted in the Chapter 4. \\

\textbf{Chapter 6: Discussion} - We discuss the overall findings and results of our experiments. Furthermore, we discuss the current capabilities of our programs, including limitations, and answer our overarching goals. Finally, we discuss both enhancing qualities of the Julia programming language, as well as current limitations within \acrshort{ai} programming. \\

\textbf{Chapter 7: Conclusion and Further Work} - This final chapter concludes our thesis. We conclude our findings and discuss further work. \\

\textbf{Overlap with Project thesis}

Parts of this thesis are based on my submitted project assignment in TDT4501 titled "Parallel DAS Processing: Julia is all you need". Section \ref{met:Judas} is a continuation of previous work conducted in the project thesis \cite{projthesis}. We provide a clear distinction of what has been done before and what's new in regard to Judas (fka. Emerald). Section \ref{met:Julia} is a summarization of an introduction to Julia, previously covered in Section 2.1 of the project thesis. Section "Reflections on Julia" in Chapter 7 covers some of the previous talking points in the project thesis, but the subsection about Julia for \acrshort{ai} programming is new.