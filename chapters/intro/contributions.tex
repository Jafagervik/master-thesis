\section{Research Questions and Contributions}
\label{intro:contribs}

The following list provides the research questions prompted by this thesis: 

\begin{enumerate}
    \item How do the spatial characteristics of \acrshort{das} data impact the performance of autoencoders for anomaly detection?
    \item How does half-precision inference affect anomaly detection in \acrshort{das} data?
    \item How can we make the entire data processing pipeline, from loading \acrshort{das} data to detecting anomalies, as fast and efficient as possible?
\end{enumerate}

In this thesis, we study processing and autoencoder-based anomaly detection using \acrshort{das} data. Our work has led to several contributions, including:

\begin{itemize}
    \item \textbf{Autoencoder based anomaly detection}: We compare the effectiveness of different autoencoders for anomaly detection on dense \acrshort{das} data. In particular, we explore the spatial characteristics of in \acrshort{das} data.
    \item \textbf{Julia for data science and \acrshort{ai}}: We evaluate Julia as a programming language for developing high-performance applications and \acrshort{ai} programming.
    \item \textbf{Software}: The following software has been produced as a part of this thesis:
    \begin{itemize}
        \item \textbf{Judas}: A software package developed in Julia for processing \acrshort{das} data. Initially introduced in our project thesis \cite{projthesis}, Judas is now further extended, improved, fully operational, and open for members of \acrshort{cgf}. 
        \item \textbf{TinyDAS}: An open-source program written in Python, specifically designed for training and evaluation of autoencoder models, as well as performing anomaly detection on \acrshort{das} data \footnote{\url{github.com/Jafagervik/TinyDAS}}. This program contains code and hyperparameters for 4 different autoencoders.  We establish how Tinygrad \cite{tinygrad} as a software package can be used to create hardware agnostic \acrshort{ai} programs that are scalable across multiple accelerators without changing source code. 
    \end{itemize}
\end{itemize}

Overall, we seek to enhance \acrshort{das} data processing and compare the effectiveness of smaller autoencoder models for anomaly detection on this data. In particular, we hope that members of \acrshort{cgf} can use and improve these tools to further \acrshort{das} research.