\section{Contributions}

In this thesis, we study processing and \acrshort{ai} based anomaly detection within \acrshort{das} data.
Our work has led to several contributions, most notably:

\begin{itemize}
    \item \textbf{Autoencoder-based anomaly detection}: We study the effectiveness of different autoencoders for anomaly detection on dense \acrshort{das} data. In particular, we explore how \acrshort{das} data can be classified as an image reconstruction problem, contrary to a time-series problem. Additionally, we discuss the effectiveness of half-precision training and inference.
    \item \textbf{Julia for datascience and \acrshort{ai}}. We evaluate Julia as a programming language for developing highly performant programs and \acrshort{ai} programming.
    \item \textbf{Software}. The following software has been produced as a part of this thesis:
    \begin{itemize}
        \item \textbf{Judas}. A software package developed in Julia for processing \acrshort{das} data. Initially introduced in our project thesis \cite{projthesis}, Judas is now fully operational but only available for members of \acrshort{cgf}. \\
        \item \textbf{TinyDAS}. An open-source program written in Python, specifically designed for training and evaluation of autoencoder models, as well as performing anomaly detection on \acrshort{das} data \footnote{\href{github.com/Jafagervik/TinyDAS}{TinyDAS Repository}}. This program contains code and hyperparameters for 4 different autoencoders.  We establish how Tinygrad \cite{tinygrad} as a software package can be used to create hardware agnostic \acrshort{ai} programs that are scalable across multiple accelerators without changing source code. \\
        \item \textbf{JudasNET}. An open-source repository with examples of autoencoders written in Julia \footnote{\href{github.com}{JudasNet}}.
    \end{itemize}
\end{itemize}

Overall, we seek to improve \acrshort{das} data processing and study the effectiveness of autoencoder models for anomaly detection on this data. In particular, we hope that members of \acrshort{cgf} can use and improve these tools to further \acrshort{das} research.