\section{Conclusion}

Importance of big data processing and DAS data processing techniques
parallelization techniques 
Contributing to open-source das research

Open-source \acrshort{das} datasets are currently limited. 


Even though we have highlighted the \textit{spatio-temporal} aspects of \acrshort{das} data, the presented models do not cover more than the spatial dimension of \acrshort{das} data. Our comparisons try to prove the need for more complex network architectures, combining both time and space dimensions of \acrshort{das} data. The computation of these networks 



\acrshort{das} signal processing is becoming increasingly more important, highlighting the potential for \acrshort{hpc}

In this thesis, we've presented several ways of dealing with the increasing amounts of \acrshort{das} data. From 

Unsupervised autoencoders still prove important  . Their ability to capture novel anomalies, as highlighted in Section \ref{back:anomdet}, 



The topics of this thesis have been broad 

In this thesis, we have covered multiple aspects of \acrshort{das} signal processing, from preprocessing of data, to

\begin{itemize}
    \item An extensive comparison betwe , further highlighting the importance of capturing spatio-temporal features of \acrshort{das} signal data
    \item 
\end{itemize}


