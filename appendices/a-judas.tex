\chapter{Judas}
\label{app:judas}


\section{Packages Used}
\label{app:jupacks}

\begin{table}[!h]
\centering
\caption{Packages used in Judas}
\label{tab:judas-packages}
\small
\begin{tabular}{>{\raggedright\arraybackslash}p{0.25\textwidth}>{\raggedright\arraybackslash}p{0.65\textwidth}}
\toprule
\textbf{Package Name} & \textbf{Description} \\
\midrule
%\rowcolor{gray!10} Flux & AI library \\
CUDA & NVIDIA CUDA programming \\
\rowcolor{gray!10} DSP & Digital Signal Processing \\
DataFrames & Working with Dataframes \\
\rowcolor{gray!10} JLD2 & Saving and loading of models \\
HDF5 & HDF5 wrapper for Julia \\
\rowcolor{gray!10} LinearAlgebra & Linear Algebra package \\
Statistics & Distributions and common functions for statistics\\
\rowcolor{gray!10} Mmap & Memory mapped I/O for working with large arrays \\
Distributed & Parallel computing in Julia \\
\rowcolor{gray!10} FFTW & FFTW wrapper for Julia \\
Dates & Datetime library \\
\rowcolor{gray!10} Plots & Plotting utilities \\
Colors & Extra colorschemes for plots \\
\rowcolor{gray!10} BenchmarkTools & Benchmark tools and utilities \\
SymPy & Julia wrapper for working with symbolic notation \\
\bottomrule
\end{tabular}
\end{table}

\section{Load DAS Files function}
\label{app:loaddas}
\lstinputlisting[label={code:loaddas},caption=Load DAS Files, language=Julia]{code/loaddas.jl}


\section{Serial Resampling}
\lstinputlisting[label={code:serialdas},caption=Serial resample function, language=Julia]{code/serialresample.jl}