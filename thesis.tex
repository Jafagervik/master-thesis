\documentclass[british,titlepage]{ntnuthesis}

\title{Artificial Intelligence and its Geophysical applications}
\shorttitle{AO and DAS data}
\author{Jorgen Aleksander Fagervik \\
        Supervisor: Ole Jakob Mengshoel}
\shortauthor{Jorgen A. F.}
\date{CC-BY \ntnuthesisdate}

\addbibresource{thesis.bib}


% From https://www.overleaf.com/learn/latex/Glossaries

\makeglossaries % Prepare for adding glossary entries

\newglossaryentry{julia}
{
        name=Julia,
        description={Is an all-purpose programming language specially suited for
scientific computing}
}

\newglossaryentry{python}
{
        name=Python,
        description={Is an all-purpose general programming language suited for scripting, data-science and web applications}
}

\newglossaryentry{llvm}
{
        name=LLVM,
        description={Low Level Virtual Machine, better known as LLVM, is a project trying to provide a modern, SSA-based compilation strategy capable of supporting both static and dynamic compilation of arbitrary programming languages \cite{llvm}}
}

\newglossaryentry{fftw}
{
        name=FFTW,
        description={Fastest Fourier Transform in the West is one of the most famous implementations of the \acrshort{dft} algorithm. It is specialized for running on \acrlong{cpu}s}
}

\newglossaryentry{relu}
{
    name=ReLU,
    description={Rectified Linear Unit is one of the most commonly used activation functions within \acrshort{dnn}s}
}

\newglossaryentry{bibliography}
{
        name=bibliography,
        plural=bibliographies,
        description={A list of the books referred to in a scholarly work,
typically printed as an appendix}
}

\newglossaryentry{maths}
{
    name=mathematics,
    description={Mathematics is what mathematicians do}
}
\newglossaryentry{pubdas}
{
    name=PubDAS,
    description={A PUBlic Distributed Acoustic Sensing Datasets Repository for Geosciences}
}

\newglossaryentry{idun}{
    name=IDUN,
    description={The Idun cluster is a project between NTNU's faculties and the IT division that aims at providing a high-availability and professionally administrated compute platform for NTNU}
}

\newglossaryentry{svm}
{
    name=Support Vector Machine,
    description={Common machine learning technique}
}




% --------------------
% ----- Acronyms -----
% --------------------

\newacronym{ntnu}{NTNU}{Norwegian University of Science and Technology}
\newacronym{ai}{AI}{Artificial Intelligence}
\newacronym{ast}{AST}{Abstract Syntax Tree}
\newacronym{mb}{MB}{Megabyte}
\newacronym{gb}{GB}{Gigabyte}
\newacronym{tb}{TB}{Terrabyte}
\newacronym{ml}{ML}{machine learning}
\newacronym{fft}{FFT}{Fast Fourier Transform}
\newacronym{rfft}{RFFT}{Fast Fourier Transform for Real Numbers}
\newacronym{dft}{DFT}{Discrete Fourier Transform}
\newacronym{mpi}{MPI}{Message-passing interface}
\newacronym{ram}{RAM}{Random-access memory}
\newacronym{gcd}{GCD}{Greatest Common Divisor}
\newacronym{hpc}{HPC}{High Performance Computing}
\newacronym{api}{API}{Application program interface}
\newacronym{gpu}{GPU}{Graphics processing unit}
\newacronym{tpu}{TPU}{Tensor processing unit}
\newacronym{cpu}{CPU}{Central processing unit}
\newacronym{das}{DAS}{Distributed acoustic sensing}
\newacronym{ann}{ANN}{Artificial neural network}
\newacronym{cnn}{CNN}{Convolutional neural network}
\newacronym{rnn}{RNN}{Recurrent Neural Network}
\newacronym{dnn}{DNN}{Deep Neural Network}
\newacronym{repl}{REPL}{Read-eval-print loop}
\newacronym{lstm}{LSTM}{Long short-term memory}
\newacronym{jit}{JIT}{Just-in-time}
\newacronym{hdf}{HDF}{Hierarchical Data Format}
\newacronym{hdf5}{HDF5}{Hierarchical Data Format version 5}
\newacronym{sisd}{SISD}{Hierarchical Data Format version 5}
\newacronym{simd}{SIMD}{Single instruction, multiple device}
\newacronym{misd}{MISD}{Multiple instructions, single device}
\newacronym{mimd}{MIMD}{Multiple instructions, multiple device}
\newacronym{spmd}{SPMD}{Single program, multiple device}
\newacronym{cgf}{NTNU CGF}{NTNU Centre for Geophysical Forecasting}
\newacronym{posix}{POSIX}{Portable Operating System Interface}
\newacronym{asn}{ASN}{ALCATEL SUBMARINE NETWORKS}
\newacronym{dsp}{DSP}{Digital Signal Processing}
\newacronym{adam}{ADAM}{Adaptive Moment estimation}
\newacronym{sgd}{SGD}{Stochastic Gradient Descent}
\newacronym{gru}{GRU}{Gated Recurrent Unit}
\newacronym{llm}{LLM}{Large Language Model}
\newacronym{sota}{SOTA}{State of the Art}
\newacronym{dl}{DL}{Deep Learning}
\newacronym{vae}{VAE}{Variational Autoencoder}
\newacronym{mae}{MAE}{Mean Absolute Error}
\newacronym{mse}{MSE}{Mean Squared Error}
\newacronym{gan}{GAN}{Generative Adverserial Network}
\newacronym{dbscan}{DBSCAN}{Density-Based Spatial Clustering of Applications with Noise}
\newacronym{adagrad}{AdaGrad}{Adaptive Gradient Algorithm}
\newacronym{fir}{FIR}{Finite Impulse Response}
\newacronym{elbo}{ELBO}{Evidence Lower BOund}
\newacronym{roi}{ROI}{Range of Interest}
\newacronym{rf}{RF}{Radio Frequency}
\newacronym{kld}{KLD}{Kullback-Leibler Divergence}

\newacronym{pr}{PR}{Precision-Recall}
\newacronym{auc}{AUC}{Area-Under-Curve}
\newacronym{cae}{CAE}{convolutional autoencoder}
\newacronym{cvae}{CVAE}{convolutional variational autoencoder}
\newacronym{fcnn}{FCNN}{Fully Connected Neural Networks}
\newacronym{foss}{FOSS}{free and open-source software}

 % add glossary and acronym lists before document
\usepackage[table,xcdraw]{xcolor}
\usepackage{siunitx}
%%
%% Julia definition (c) 2014 Jubobs
%%
\lstdefinelanguage{Julia}
  {morekeywords={abstract,break,case,catch,const,continue,do,else,elseif,
      end,export,false,for,function,immutable,import,importall,if,in,
      macro,module,mutable,otherwise,quote,return,switch,true,try,type,typealias,using,while,@btime},
   otherkeywords={
        <, >, ->, <=, >=, ==, ===, !==, !=
   },
   numbers=left,
   numberstyle=\scriptsize,
   sensitive=true,
   alsoother={\$},
   morecomment=[l]\#,
   morecomment=[n]{\#=}{=\#},
   morestring=[s]{"}{"},
   morestring=[m]{'}{'},
}[keywords,comments,strings]

\lstset{
    language         = Julia,
    basicstyle       = \ttfamily,
    keywordstyle     = \bfseries\color{blue},
    stringstyle      = \color{magenta},
    commentstyle     = \color{teal},
    showstringspaces = false,
}



\begin{document}

\input{chapters/0c-foward.tex}
\chapter*{Abstract}

Signal processing and \acrfull{ai} has become .

Parts of this thesis are taken from or based on my submitted project assignment in the subject TDT4501 with the title "Parallel DAS Processing: Julia is all you need".
\chapter*{Sammendrag}

Denne avhandlingen undersøker parallell prosessering av distribuert akustisk sensing (DAS)-data og anvendelsen av autoencodere for anomalideteksjon på disse dataene. Forskningen adresserer det økende behovet for effektiv prosessering og analyse av storskala \acrshort{das} data, med potensielle anvendelser som spenner fra jordskjelvdeteksjon til overvåking av jordskred. \\

Vi presenterer to nye programmer: 1) Judas.jl, en Julia-pakke for forbehandling av store volumer av \acrshort{das} data, og 2) TinyDAS, et Python-program for trening av flere autoencodere på tvers av flere GPU-er, og utføring av anomalideteksjon på DAS-data. Disse verktøyene har som mål å automatisere deteksjonen av anomalier i sanntids datastrømmer, og potensielt redusere behovet for manuell intervensjon og forbedre nøyaktigheten av lignende dataanalyse. \\

Vår metodikk involverer validering av disse programmene på både proprietære og åpen kildekode DAS-datasett fra den virkelige verden. Resultatene demonstrerer skalerbare løsninger for både dataprosessering og anomalideteksjon, og viser betydelige forbedringer i effektivitet og nøyaktighet sammenlignet med eksisterende metoder ved NTNU Senter for Geofysisk Prediksjon (CGF). \\

Denne forskningen bidrar til feltet DAS-dataanalyse ved å tilby robuste verktøy for håndtering av storskala DAS-data og utnyttelse av ulike effektive autoencodere for anomalideteksjon. Funnene har implikasjoner for ulike industrier som benytter DAS-teknologi, inkludert NTNU CGF, og tilbyr potensielle forbedringer i dataprosesseringspipelines og anomalideteksjonskapabiliteter. \\

Deler av denne avhandlingen er hentet fra eller basert på min innleverte prosjektoppgave i faget TDT4501 med tittelen "Parallel DAS Processing: Julia is all you need".



\tableofcontents
\listoffigures
\listoftables
\lstlistoflistings

\printglossary[type=\acronymtype] % Print acronyms
\printglossary                    % Print glossary

\chapter{Introduction}
\label{chap:introduction}

In this very first chapter, we set up the rest of this thesis. We cover our moitvation for this projects, our goals and research questions, what contributions this paper have given as well as an outline for this thesis.

\section{Motivation and Challenges}

\subsection{Context} 

\acrfull{das} is a rather new technology that allows for real-time analysis over fiber-optical cables. This technology has gained more recognition within the last decade, and due to their high sensitivity, \acrshort{das} systems can detect subtle environmental changes and anomalies. Analyzing these irregularities is a common and crucial task in various fields, and it can be applied to tasks spanning landslide and earthquake detection as well as railroad and maritime monitoring. The ability to process and interpret \acrshort{das} data effectively is essential for extracting meaningful insights from these complex measurements. \\

\begin{figure}[!h]
    \centering
    \includegraphics[width=0.7\linewidth]{figures/das.png}
    \caption{Showcase of how \acrshort{das} signals are recorded}
    \label{fig:das-fig}
\end{figure}

Traditionally, clustering-based \acrfull{ml} techniques such as K-MEANS \cite{hartigan1979k} DBSCAN \cite{ester1996density} have been quite popular for anomaly detection \cite{anomaly}. Across the last years, a popular modification to the DBSCAN algorithm, HDBSCAN \cite{rahman2016hdbscandensitybasedclustering}, has also shown prowess in clustering-based anomaly detection \cite{ariyaluran2022clustering}. However, these methods often require manual feature engineering, require labeled datasets, or generally do not scale to large datasets. 

\begin{figure}[!h]
    \centering
    \includegraphics[scale=0.4]{figures/anolay_line.png}
    \caption{Example of anomalies in a time series}
    \label{fig:anomaly_example}
\end{figure}

\acrshort{das} technology in itself has now started garnering attention for research, and several papers have previously studied how one can process this data. \acrshort{ai} and \acrshort{ml} models have been constructed for looking at time series data and analyzing sensor data, although several of these have been studied.  Only recently has \acrshort{ai}


\subsection{Motivating challenges}. 

\acrfull{cgf} spend a lot of time and resources on processing and analyzing \acrshort{das} data. Current tools for processing are quite slow and do not utilize parallelization techniques that have the potential to drastically speed up computations. Additionally, analysis often uses more traditional signal processing techniques, not leveraging the potential benefits of more novel \acrshort{ann} methods. \\ 


Recorded \acrshort{das} data has the potential to become quite large, up to several terabytes per experiment, underlying the importance of efficient algorithm design and processing techniques. Generally, languages such as Python and Matlab are used for DAS analysis due to their framework for data science applications. However, these programming languages are not designed for data-intensive applications without having to leverage languages such as C. Julia is a more novel language aimed at both data science and in general \acrfull{hpc} applications, and could prove really powerful as an alternative to an existent program.


However, with the upcoming of \acrshort{das}, both unsupervised and supervised \acrfull{dl} methods have proven to produce even better results for anomaly detection. For \acrshort{das} data specifically, both scalability and manual labeling can become quite tedious or outright non-feasible. For 

In later years, unsupervised learning has returned after the explosion of generative models [CITE]. Compared to their supervised alternatives, unsupervised do not require manual labeling. They're, therefore, not prone to some of the more common problems within supervised methods, such as detecting irregular events.   are well suited for detecting novel anomalies \cite{wei2022lstmautoencoder, srivastava2016unsupervised} compared to its supervised alternatives, and do not require manual labeling. This makes them 

Current autoencoder-based approaches to anomaly detection of \acrshort{das} do not emphasize the overall memory consumption or the conversion of models to a real-time environment. This 



Previous work on this data \cite{projthesis} revolved around processing \acrshort{hdf5} files as fast and efficiently as possible, trying to parallelize already existent code, and take advantage of newer technologies, such as Julia.

\section{Goals}

Our goals for this thesis are as following: 

\begin{enumerate}
    \item Find out if Julia is a well suited  language when it comes to big data and \acrshort{ai}.
    \item What kinds of unsupervised models can work well with \acrshort{das} data.
    \item If our tool can be efficiently used by other members at \acrshort{cgf}.
\end{enumerate}


\subsection{Research Questions}

In addition to our goals, the following are a set of questions we want answers to by the end of the article

\section{Contributions}

(Note to Ole)
Goal: 
    1. Develop, or improve tools that can process \acrshort{das} data and detect anomalies, both opensource and for CGF
    
RQ:
    1. Is Float16 training sufficient for training \acrshort{das} data in the context of data reconstruction and anomaly detection
    2. How does the different autoencoders compare, and can we make a convolutional variational autoencoder for das data

In this thesis, we study processing and autoencoder-based anomaly detection within \acrshort{das} data. Our work has led to several contributions, including:

\begin{itemize}
    \item \textbf{CVAE}: We present a \textit{novel} \acrfull{cvae} model for anomaly detection on \acrshort{das} data.
    \item \textbf{Autoencoder-based anomaly detection}: We compare the effectiveness of different autoencoders for anomaly detection on dense \acrshort{das} data. In particular, we explore anomaly detection on \acrshort{das} data as an image reconstruction problem, contrary to a time-series problem. Additionally, we discuss the effectiveness of half-precision training and inference.
    \item \textbf{Julia for datascience and \acrshort{ai}}: We evaluate Julia as a programming language for developing highly performant programs and \acrshort{ai} programming.
    \item \textbf{Software}: The following software has been produced as a part of this thesis:
    \begin{itemize}
        \item \textbf{Judas}: A software package developed in Julia for processing \acrshort{das} data. Initially introduced in our project thesis \cite{projthesis}, Judas is now fully operational but only available for members of \acrshort{cgf}. 
        \item \textbf{TinyDAS}: An open-source program written in Python, specifically designed for training and evaluation of autoencoder models, as well as performing anomaly detection on \acrshort{das} data \footnote{\url{github.com/Jafagervik/TinyDAS}}. This program contains code and hyperparameters for 4 different autoencoders.  We establish how Tinygrad \cite{tinygrad} as a software package can be used to create hardware agnostic \acrshort{ai} programs that are scalable across multiple accelerators without changing source code. 
        \item \textbf{JudasNET}: An open-source repository with examples of autoencoders written in Julia \footnote{\url{github.com/Jafagervik/JudasNET}}.
    \end{itemize}
\end{itemize}

Overall, we seek to improve \acrshort{das} data processing and compare the effectiveness of autoencoder models for anomaly detection on this data. In particular, we hope that members of \acrshort{cgf} can use and improve these tools to further \acrshort{das} research.
\section{Thesis outline}

The following list is an outline over the rest of the thesis, and what will be presented for each section. \\

\textbf{Chapter 1: Introduction} - We present the problems, what we want to find out and our motivation for this project. \\

\textbf{Chapter 2: Background and Theory} - We go more in dept about theory regarding both relevant \acrshort{dl} architectures and their applications to our problems, as well as some introduction to applicable signal processing techniques.  \\

\textbf{Chapter 3: Literature Review} - Here we discuss relevant literature, both on \acrshort{ai} for signal processing in general, but also around \acrshort{das} data. \\

\textbf{Chapter 4: Method} - We cover all our practical work, and implementation decisions. This includes data processing, \acrshort{api} design, network architecture and experiments. \\

\textbf{Chapter 5: Results and Discussions} - We present our findings, compare results, discuss different outcomes and about Julia in general. \\

\textbf{Chapter 6: Conclusion and Further Work} - This final chapter concludes our findings. We answer the questions asked in \textbf{Chapter 1}, and try to see where all this leaves us going forward. \\


\chapter{Background and Theory}
\label{chap:back}

In this chapter, we discuss the underlying theory necessary for our work. That includes \acrshort{ai} and \acrshort{ml} theory, as well as relevant techniques within signal processing. \\

\section{Julia}

Julia is a high performance, dynamically typed programming language created by MIT in 2012, officially released in 2014. It has developed itself into a real alternative to Python, R and MatLab, while outperforming all on them on general benchmarks. As mentioned in \cite{projthesis}, this is mainly due to how Julia is a "Just ahead of time" compiled language. \\ 

Julias fluent type system, accompanied by easy syntax, high performance, \acrshort{repl} tools makes it a great contender for data analysis. We've previously proven how Julia effectively deals with I/O operations, .


DOI \cite{doi:10.1137/141000671}

FLUXXERT
\section{\acrshort{das} and Digital Signal Processing Techniques}
\label{back:dsp}

\subsection{Distributed Acoustic Sensing Data}
\label{back:das}

\acrshort{das} data can be interpreted as a multi-sensor time series, where each channel (sensor) stores signal values for different sample times. THis data can be stored in different file formats, but due to the hierarchical nature of the data, formats such as TDMS \cite{10.1145/800196.805973} or \acrshort{hdf5} \cite{koranne2011hierarchical} are commonly used \cite{spica2022pubdas}. Their hierarchical nature is ideal for complex datasets, which often require additional metadata. Regardless of the chosen format, certain metadata are crucial for effectively handling and interpreting these data, including:
\begin{itemize}
    \item \textbf{Gauge length} is the spatial resolution of measurements.
    \item \textbf{Channel distance} stores information on spatial sampling. Not all channels along the total measurement are stored, so to understand the location of a signal, the gauge length, combined with the channel distance, tells us the exact distance from the start of the measurement.
    \item \textbf{Sample Rate}, or sample frequency, is the temporal resolution of the data and is measured in hertz. 
\end{itemize}
%
The \textit{spatio-temporal} aspects of \acrshort{das} come from how the data is represented. This data can be represented as a one-channel image, as shown by the seismic heatmap in Figure \ref{fig:dasframe-ex}. 
%
\begin{figure}[!h]
    \centering
    \includegraphics[width=0.7\linewidth]{figures/das_example.png}
    \caption{\textbf{Visualization} of normalized \acrshort{das} data as a heatmap. The vertical axis represents time (increasing downwards), while the horizontal axis shows a little more than 2000 spatial channels along the fiber. Color intensity indicates the strain rate, with reds representing higher rates and blues lower rates. The diagonal patterns in the middle likely represent propagating seismic waves or other dynamic strain events detected.}
    \label{fig:dasframe-ex}
\end{figure}
%
\subsubsection{Column- and Row-major Memory Alignment}
%
When working with large matrices, such as \acrshort{das} data, the order of the axis is important due to how different compilers access memory. When fetching a variable $x$, the \acrfull{cpu} will try to fetch a  \textit{cache-line}, and depending on the memory layout, this affects performance when iterating over the matrix. How a programming language or compiler organizes data in memory can significantly impact performance, especially for large-scale computations. The two types of memory alignment are \textit{column-major} and \textit{row-major} as shown in Figure \ref{fig:rowcol}.
%
\begin{figure}[!h]
    \centering
    \includegraphics[width=0.5\linewidth]{figures/rowcol.png}
    \caption{Row-major (Left) and column-major (Right) memory ordering}
    \label{fig:rowcol}
\end{figure}
%
In the case of \acrshort{das} data, where calculations are often performed on a per-channel level, a language like Julia, MatLab, or Fortran would benefit from storing each channel in a column. This alignment can lead to several advantages, including:
\begin{enumerate}
\item \textbf{Improved cache utilization:} When processing data channel by channel, column-major storage ensures that the data for each channel is contiguous in memory, reducing cache misses.
\item \textbf{Reduced memory fragmentation:} Storing long time series for each channel in columns can lead to better memory allocation and less fragmentation.
\item \textbf{Vectorization opportunities:} Many modern processors support \acrfull{simd} operations, which can be more efficiently applied to contiguous data \cite{ren2006optimizing}.
\end{enumerate}
%
\subsection{Radio Frequency Filtering}
%
\acrfull{rf} filtering is of paramount importance in \acrshort{das} and \acrfull{dsp}. The signal quality can be improved by removing unnecessary noise from the data, and can decrease the overall of signal loss. In general, there are four types of filters, and can be defined as follows:
%
\begin{itemize}
    \item \textit{Band-pass filters} only allows frequencies between two cutoff frequencies $F_{low}$ and $F_{high}$
    \item \textit{Band-stop filters} stops frequencies between two cutoff frequencies $F_{low}$ and $F_{high}$
    \item \textit{Low-pass filters} only allows frequencies above the cut-off frequency $F_{low}$
    \item \textit{High-pass filters} only allows frequencies above the selected frequency $F_{high}$
\end{itemize}
%
A common approach to preprocessing \acrshort{das} data usually involves applying a bandpass to the signal matrix. Due to \acrshort{das} data being sensitive and capturing a broad range of frequencies, limiting the signals to a range of interest, depending on domain and application is beneficial.

\vspace{0.5cm}

\begin{figure}[!h]
    \centering
    \includegraphics[width=0.8\linewidth]{figures/lowhighpass.png}
    \caption{Low-, High- and Band-pass filters}
    \label{fig:rffilters}
\end{figure}
%
\subsection{Tukey Window}
\label{dsp:tukey}

Window functions are functions often used in \acrshort{dsp} to avoid artifacts. This is done by setting values outside a pre-defined interval to zero and applying a taper from the passband to the first zero value. The Tukey window \cite{tukey1967introduction}, also known as the \textit{cosine-tapered window}, is a common approach to reducing edge effects, and can be formulated as such:

\[
    w(x)= 
\begin{cases}
    \frac{1 + \cos{2 \pi \alpha (x + \frac{1-\alpha}{2})}}{2}, & \text{if } x \leq \frac{1-\alpha}{2}\\
    1,              & \text{if } \frac{\alpha}{2} < x \leq \frac{\alpha}{2}\\
    \frac{1 + \cos{2 \pi \alpha (x - \frac{1-\alpha}{2})}}{2}, & \text{if } x > \frac{1-\alpha}{2}
\end{cases}
\]
where a higher $\alpha$ leads to a smoother cutoff in the output.
%\begin{figure}[!h]
%    \centering
%    \includegraphics[width=0.6\linewidth]{figures/tukey_windows_high_res.jpg}
%    \caption{Tukey window across different $\alpha$ values. The window becomes a rectangle when $\alpha = 0$}
%    \label{fig:tukeywindow}
%\end{figure}
\subsection{Resampling}
%
''Resampling methods are statistical procedures that reuse the sample data for the purpose of statistical inference''\cite{https://doi.org/10.1002/widm.1054}. In many applications, data is often initially collected at a very high sampling rate to capture fine details. However, this high sampling rate is not computationally efficient or even necessary for all analytical purposes. Down-sampling, also referred to as decimation within the \acrshort{dsp} field, is a form of resampling where the sampling rate is reduced, can be applied to:
%
\begin{itemize}
    \item Decrease memory consumption for data storage
    \item Reduce computational time for data processing
    \item Balance the trade-off between processing efficiency and data resolution
\end{itemize}
%
By reducing the sampling rate, one may retain the essential characteristics of the data while reducing data volume and computational requirements. However, one must be careful to avoid aliasing and ensure that the resampled data accurately represents the phenomena of interest. For \acrshort{das} data, this can differ drastically from one experiment to another. 

\subsection{Unsupervised Learning}

The major bottleneck of all kinds of machine learning tecniques is data. The more diverse and varied a .

When it comes to \acrshort{ai} and \acrshort{ml} we usually differentiate between tree major types, those being supervised, unsupervised and semi-supervised learning (or self-supervised learning). They differ in the roles that can occur.

Unsuprtwised 


\section{Anomaly Detection}

Anomaly detection is all about finding outliers in datasets, often referred to as standard deviations. 

Given the set $X = \{x_1, x_2, ..., x_n\}$, we define an anomaly as any value $\theta$ in set $X$ such that , where $\epsilon$ is based on a heuristic.


Anomaly detection is used in a plethora of fields, such as networking, medical informatics and more. 

For \acrshort{das} data specifically, anomaly detection can be used for detecting clusters of signals that don't correspond to the predispused target feature. Registering these outlier signals and receiving real time information about these could prove vital in some cases,  and in best case scenario save lives.

\subsection{Multivariate Anomaly detection}

In a one-dimensional time series, finding anomalies tend to be rather trivial. After taking neighbor values into account, look for sever outliers, .

Multivariate data analysis refers to statistical lookings from two or more variabels. In the case of sensor data, one often look at multiple sensors. 

Given a matrix $a$ of data:

\begin{equation}
\centering
\begin{matrix}
a_{11} &  0      & \ldots & a_{1n}    \\
0      &  a_{22} & \ldots & a_{2n}    \\
\vdots & \vdots  & \ddots & \vdots \\
a_{m1} &  0      &\ldots & a_{mn}
\end{matrix}
\caption{Matrix of data}
\end{equation}

We are interested in finding a region $a_{ij} to a_{kl}$ where $i < k \And j < l$ st. the values within these regions falls outisde of the general range
\section{Recurrent Neural Networks}

\acrfull{rnn}

The main advantage of \acrshort{rnn} 

The building blocks of an \acrshort{rnn} is the recurrant neuron. They take an input $x_t$ at time $t$, with the state at previous time step $h_{t-1}$ and produces the next time step. 

\begin{equation}
    h_t = f(W_hh_{t-1}+W_xx_t)
\end{equation}

RNN have been used witihn geophysical applications before \cite{maulik2020recurrent}. 

Typically, \acrshort{relu} is used as the activation function in these networks, as it is cheap to perform and we don't need more expressive activation functions. Other common activation functions are sigmoid and tanh. 

$$g(z) = max(0, z)$$
$$g(z) = \frac{1}{1 + e^{-z}}$$
$$g(z) = \frac{e^z - e^{-z}}{e^z + e^{-z}}$$

Although RNN has several advantages, they are highly prone to both vanishing gradients as well as exploding gradients. 

\subsubsection{LSTM - Long Short Term Memory}

One of two more common solutions to avoid the pitfalls \acrshort{rnn}s give us, is to use \acrshort{lstm} cells. 

 As we've mentioned, regular \acrshort{rnn}s struggle with long term memory, something the \acrshort{lstm} cells solves. \\ 


An \acrshort{lstm} cell is built up by the following components: an input gate $i$, a forget gate $f$, a candidate state $g$, an output gate $g$ .
These cells traditionally make use of the sigmoid non-linearity function $\sigma()$

\begin{figure}[h]
    \centering
    \includegraphics{figures/lstmcell.png}
    \caption[scale=0.4]{Example of an LSTM cell}
    \label{fig:lstmcell}
\end{figure}

Compared to the equation for forward pass 

\begin{equation} \label{eq:cnn}
    Y_t = W_xx_t + b
\end{equation}

the \acrshort{lstm} keeps the previous state in mind, thus giving us the equation


\begin{equation} \label{eq:lstm}
    Y_t = W_hh_{t-1} + W_xx_t + b
\end{equation}

Adding batch normalization to this equation and we end up with the following equation \cite{cooijmans2017recurrent}: 

\begin{equation} \label{eq:bnlstm}

c_t = \sigma(\Tilde{\textbf{f}_t} \cdot c_{t-1} + \sigma(\textbf{\Tilde{i}}_t) \cdot tanh(\Tilde{g}_t)
h_t = \sigma(\Tilde{o}_t) \cdot tanh(BN(\textbf{c_t}; \gamma_c, \beta_c))
    
\end{equation}

By applying batch normalization to \acrshort{lstm}s, not only did the models converge faster, the performance was up to par with the unnormalized \acrshort{lstm} \cite{cooijmans2017recurrent}.

Whereas \acrshort{rnn}s shines at \acrshort{nlp}, speech recognition, and media processing, \acrshort{lstm}s is vastly better for time series forecasting due to the memory gates. This makes \acrshort{lstm}s a suitable option for us when working with anomaly detection on sensor data, which in its essence is nothing more than more complex time series data across multiple columns or channels. 

The alternative to using \acrshort{lstm} nodes for our network would be \acrfull{gru}s.


\subsection{Autoencoder}

The autoencoder is a type of network used to learn efficient encodings of unlabeled data. 

Autoencoders are split into two parts. The encoder $E_\phi$ and the decoder $D_\theta$. The relationship between these can be articulated as such: 

\begin{figure}[h]
    \centering
    \includegraphics[scale=0.5]{figures/ae.png}
    \caption{Autoencoder Architecture Diagram}
    \label{fig:aediagram}
\end{figure}

\begin{equation}
E_\phi: X \rightarrow Z 
\end{equation}

\begin{equation}
D_\theta: Z \rightarrow X
\end{equation}

The optima for any kind of autoencoder becomes that of lossless encoding, which can further be described as such:

\begin{equation}
    X = D_\theta(E_\phi(X))
\end{equation}

When a auto encoder is trained to max effiency, we can in some cases remove the decoder part. Since the goal in the beginning was to map data to a lower-dimensional latent space, which has been increased. If ones goal is feature extraction, the decoder is not needed any more. Additionally, by removing the decoder, the overall complexity and size of the model $M$ decreases.


Common use cases for autoencoders are signal analysis, anomaly detection, reconstructing images and several other applications. 
One of the more well known usages for autoencoders

\subsubsection{Loss functions}

\textbf{\acrfull{mse}}

The \acrshort{mse} loss function, also referred to as L1 loss, is one of the most well known loss functions. It punishes bigger differences by squaring the difference between two elements in the prior and the posterior.

\begin{equation}
    MSE = \dfrac{1}{n}  \sum_{i=1}^{n}(x_i-\hat{x}_i)^2
\end{equation}

\textbf{\acrfull{mae}}

The \acrshort{mae} loss function, also referred to as L2 loss, is quite similar to the L1 loss. The difference is that the \acrshort{mae} function will only return the absolute value of the difference between two distributions, thus not caring about the difference itself.

\begin{equation}
    MAE = \dfrac{1}{n}  \sum_{i=1}^{n}|x_i-\hat{x}_i|
\end{equation}

\textbf{Hinge Loss}

\begin{equation}
    L = \max(0, 1 - x \cdot \hat{x}))
\end{equation}

\textbf{Log Loss, VAE}

The log loss, also known as Binary Cross Entropy loss

\begin{equation}
L(x, \hat{x}) = - \frac{1}{N} \sum_{i=1}^{N} \left( x_i \log(\hat{x}_i) + (1 - x_i) \log(1 - \hat{x}_i) \right)
\end{equation}


\textbf{Activation Functions}
\begin{equation}
    ReLU(z) = max(0, z)
\end{equation}

\subsubsection{Variational Autoencoder (\acrshort{vae})}

Similar to the regular autoencoder, a \acrfull{vae} also aims to map input over to a feature representation. The diffre. Broadly speaking, the difference between a \acrshort{ae} and a \acrshort{vae} is that a AE maps input to points, where as VAEs map over to a distribution in the latent space. The encoder $E$ outputs to vectors. 

Thus, the \acrlong{vae} can be seemed as a generative model.


\begin{figure}[h]
    \centering
    \includegraphics[scale=0.5]{figures/vae.png}
    \caption{Variational Autoencoder Architecture Diagram}
    \label{fig:vaediagram}
\end{figure}


VAEs can be trained by backpropagation due to something known as the \textit{reparametrization trick}. We need this because since VAEs maps to a stochastic variable, abckpropagation would else not be feasible.

\begin{align*}
\text{Given:} & \quad \text{Encoder LSTM outputs: } h \\
& \quad \text{Mean vector: } \mu \\
& \quad \text{Log variance vector: } \log(\sigma^2) \\
\text{Sample:} & \quad \epsilon \sim \mathcal{N}(0, 1) \\
\text{Reparameterization:} & \quad z = \mu + \sigma \odot \epsilon \\
\text{Decoder Input:} & \quad z \quad \text{(sampled latent vector)} \\
\text{Training Objective:} & \quad \text{Minimize reconstruction error} \\
& \quad \text{and KL divergence between } q(z|x) \text{ and } p(z) \\
\text{Loss Function:} & \quad \mathcal{L} = \text{reconstruction\_loss} + \text{KL\_divergence}
\end{align*}


\textbf{KEYS TO GETTING A GOOD VARIATIONAL AUTO ENCODER}

\begin{itemize}
    \item Pick the righ size for the latent space
    \item Learning rate scheduler  and hyperparam tuning 
    \item BETA COEFFICIENT
\end{itemize}

An effort has been made into trying to improve autoencoders for anomaly detection \cite{tan2023improving}

THIS ONE IS HIGHLY RELEVANT AND HAS MANY METRICS \cite{s23021009}

\chapter{Method}
\label{chap:method}

We start of at the same spot we left of after the thesis. Before we go on to explain what's been done this semester, lets recap. 

We started off by just getting a drive with lots of \acrshort{das} data, and a script from \acrfull{asn} which set up us perfectly. We chose to rewrite python code to a language similar to that of MAtlab and Python. We landed on Julia, and with its broad ecosytstem it has support for all kinds of development we'd ever need. \\

We first wrote a 1:1 copy of the script, and then parallellized parts of the code until we had a version that could handle large amounts of data in parallel. We also wrote methods for being able to run a window over and 


\subsection{Overview}

\subsection{SignalProcessing.jl}
\chapter{Results}
\label{chap:results}

Apples in my asshole \cite{projthesis}
\input{chapters/5-discussion}
\chapter{Conclusion}
\label{chap:conclusion}

dkl

\section{Further work}
\label{conc:further}

While our work has addressed several key challenges, both Judas and TinyDAS are in their first iterations. Further research and development can address the current limitations discussed in this thesis.

\subsection{Judas and \acrshort{das} processing}

While Judas serves the task of loading and processing \acrshort{das} data, the following list provides an overview of further advancements that can enhance its computational effectiveness, specifically targeting file loading:

\begin{itemize}
    \item Design and implement a more course-grained approach to our current solutions for finding and loading \acrshort{das} files. 
    \item Designing a parallel version of the column-wise cumulative sum function, potentially using algorithms like parallel prefix sum \cite{harris2007parallel}, where \acrshort{gpu} acceleration could be introduced. 
    \item Implementing a more flexible metadata handling system to accommodate datasets from other sources besides BANENOR.
    \item Implementing more advanced denoising and signal processing techniques, potentially utilizing denoising autoencoders \cite{eage:/content/journals/10.1111/1365-2478.13355}.
\end{itemize}

\subsection{TinyDAS and Autoencoder-based Anomaly Detection}

As discussed in Section \ref{disc:tinydas}, we have succeeded in many of our goals with TinyDAS. However, as the poor results from the variational autoencoders show, there is still much room left for tuning current models. The following list presents multiple avenues for further development and research:

\begin{itemize}
    \item Expand support for more different architecture and compare simpler models with hybrid models that combine both the temporal and spatial aspects of \acrshort{das} signals, such as CNN-LSTM architectures, or even transformer-based ones, to capture finer details.
    \item  Half-precision training has been a heavy area of attention. For now, TinyDAS supports half-precision training, inference, loss-scaling, and gradient clipping. Further improving on these to ensure gradients are within range is an important aspect of reducing runtime and memory consumption. Comparing this approach to mixed-precision could prove beneficial.
    \item Expanding support for other \acrshort{das} datasets, both from PubDAS and other public sources. This could be achieved through a more customizable format of our DataLoader class. Furthermore, implementing functionality for downloading \acrshort{das} datasets directly from the internet would lower the user entry barrier.
    \item Implement support for reading real-time datastreams, to further analyze how different models perform in real-world scenarios
\end{itemize}

\subsection{Final Remarks}

We want to thank \acrfull{cgf} for providing data and computational resources. We hope both these programs can help all their current and future members further the field of \acrshort{das} research.

\chapter*{\bibname}
\printbibliography[heading=none]

\input{chapters/papers.tex}

\appendix
\input{appendices/a-appendix.tex}

\end{document}
